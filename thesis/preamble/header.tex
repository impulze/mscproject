%----------------------------------------------------------------------
% HEADER
% Einstellungen, Makros etc.
%----------------------------------------------------------------------
\documentclass[12pt,a4paper,twoside,scrartcl,report]{thesis}
\NeedsTeXFormat{LaTeX2e}[1994/06/01]

% Verwendete Pakete
\usepackage{ae}
\usepackage{times}
\usepackage[ngerman]{babel}					% Deutsche Besonderheiten (neue Rechtschreibung)
\usepackage[utf8]{inputenc}				% Latin-1 (z.B. ß)
\usepackage[T1]{fontenc}						% T1-Schriften verwenden (statt CM)
\usepackage{textcomp}								% Zusätzliche Textsymbole von T1
\usepackage{makeidx}								% Index
\usepackage{fancyhdr}								% Definition von Kopf-/Fußzeilen
\usepackage{psboxit}
\usepackage{float}									% Gleitumgebungen
\usepackage{setspace}
\usepackage{doc}										% Für's BibTeX-Logo (Siehe vorlage.tex)
\usepackage{changebar}
\usepackage{listings}								% Formatierungen für Quellcode
%\usepackage{color}
\usepackage{multirow}
\usepackage{longtable}							% Tabellen über mehrere Seiten
\usepackage{multicol}								% Mehrspaltiger Satz
\usepackage[german]{varioref}				% Variable Referenzen
\usepackage{lscape}									% Querformat
\usepackage{footnpag}								% Fußnoten: Nummerierung auf jeder Seite bei 1 beginnen
\usepackage[normalem]{ulem}					% Unterstreichung
\usepackage{xspace}
\usepackage[bottom,hang,marginal]{footmisc}
\usepackage{amsmath}
\usepackage[bf,SL,BF]{subfigure}
\usepackage{gastex}
\usepackage{array}
\usepackage{eurosym}
\usepackage{ragged2e}
\usepackage{url}

\sloppy

% Die Algorithmus-Umgebung für Pseudocode
\usepackage{algorithmic}
\usepackage{algorithm}
\numberwithin{algorithm}{chapter}
\renewcommand{\listalgorithmname}{Verzeichnis der Algorithmen} 
\renewcommand{\algorithmiccomment}[1]{// #1}
\floatname{algorithm}{Algorithmus}
\newcommand{\theHalgorithm}{\arabic{algorithm}}
%\setlength{\footnotemargin}{0pt}

% set font-style to computer modern sans-serif (cmss)
%\renewcommand{\sfdefault}{cmss}
%\renewcommand{\familydefault}{\sfdefault}

%----------------------------------------------------------------------
% Besondere Einstellungen für PDF-Ausgabe.
% Es wird eine Fallunterscheidung getroffen und entsprechende Pakete
% eingebunden und Einstellungen vorgenommen.

% PDF oder DVI? Es wird ein Flag gesetzt.

% PDF Einstellungen
\RequirePackage{ifpdf}
\ifpdf
  \usepackage{graphicx}
  \pdfcompresslevel=9
  \usepackage[
  		pdftex, 
  		colorlinks=true,
  		linkcolor=blue,
  		urlcolor=blue,
  		citecolor=blue,
  		plainpages=false,
  		pdfpagelabels,
  		bookmarksnumbered=true,
			pdftitle={Thema der Arbeit eintragen!},
			pdfauthor={Max Mustermann},
			pdfkeywords={Key, Words} ]{hyperref}
  % PDF-Seite bei LScape-rotate drehen
  \makeatletter
  \@ifundefined{pdfpageattr}
    {}{\g@addto@macro{\landscape}{\pdfpageattr{/Rotate 90}}}
  \makeatother
% DVI (oder andere Ausgabe) Einstellungen
\else
  \usepackage[dvips]{graphicx}		% Graphikunterstützung, auch JPG und PNG
\fi

% Erlaubt das Einbinden von JPGs in das Dokument (aus Paket [dvips]{graphicx}).
% Die JPGs werden zuvor nach EPS konvertiert. Die Größe (BoundingBox)ist in <datei>.bb
% festgelegt. Diese muß manuell durch ebb.exe erzeugt werden.
\DeclareGraphicsRule{.jpg}{eps}{.bb}{}

%-----------------------------------------------------------------------
% Seitenlayout festlegen
\voffset-1in
\hoffset-1in
\setlength{\oddsidemargin}{4cm}
\setlength{\evensidemargin}{2cm}
\topmargin15pt
\textwidth150mm
\textheight230mm
\footskip1.5cm
\headheight25pt

\pagestyle{fancyplain}

% Definition von Kopf- und Fußzeilen
\lhead[\fancyplain{}{\nouppercase{\sl\rightmark}}]{\fancyplain{}{\nouppercase{\sl\leftmark}}}
%\rhead[\fancyplain{}{\nouppercase{\sl\leftmark}}]{\fancyplain{}{\nouppercase{\sl\leftmark}}}
\rhead[\fancyplain{}{\nouppercase{\sl\leftmark}}]{\fancyplain{}{\nouppercase{\sl\rightmark}}}
%-----------------------------------------------------------------------
% Definitionen für Quellcodes/Listings
\lstloadlanguages{Java}
\lstset{
  language=[AspectJ]Java,						% Java with AspectJ-Dialect
	tabsize=4,												% Tabulatorbreite
	linewidth=\linewidth,							% Width of a line
	breaklines=true,									% Break long lines
	breakatwhitespace=true,						% Only break at whitespaces
	basicstyle=\scriptsize\ttfamily,	% Schriftart/-größe	
	numbers=left,											% Linenumbers left
	numberfirstline=false,						% Not: Always number 1. line
	numberstyle=\scriptsize,					% Größe der Zeilennummern
	stepnumber=1,											% Jede 2. Zeilennummer anzeigen
	numbersep=5pt,										% Abstand Nr - Quellcode
	showspaces=false,									% Spaces nicht anzeigen
	showtabs=false,										% Tabs nicht anzeigen
	showstringspaces=false,						% Don't show tabs in strings
	showlines=false,									% Leerzeilen am Sourceende weglassen
	extendedchars=true,								% ASCII-Zeichnen > 127 zulassen 
	identifierstyle=\bfseries,					% Identifier
	keywordstyle=\bfseries,						% Keywords
	commentstyle=\itshape,						% Style of comments
	stringstyle=\ttfamily,						% Strings (!= Keywords)
	flexiblecolumns=false,						% Use fixed width for fonts
	fontadjust=true,									% "Base width" nicht jede Zeile anpassen
	frame=trbl,												% Frame; trBL
	captionpos=b,											% Position of the caption
	aboveskip=25pt,										% Space between text and the top of the listing
}

%-----------------------------------------------------------------------
%VERZEICHNISSE 

\makeindex

% Namen für Quellcodes und Quellcode-Verzeichnis
\renewcommand\lstlistingname{\normalsize Quellcode}
\renewcommand\lstlistlistingname{Verzeichnis der Quellcodes}

% Layout für Literaturverzeichnis (BibTeX)
\bibliographystyle{alphadin}			% Alphab., Verfasser + Jahr (DIN 1502)

% Inhaltsverzeichnis
\setcounter{tocdepth}{3}
\setcounter{secnumdepth}{3}

\makeatother

%-----------------------------------------------------------------------
% Debug-Optionen - sollten fuer die Final Version gelöscht werden
\vrefwarning
\nochangebars

% Tabellen über Seitengrenzen zulassen
\setlongtables

% Absatztrennung durch Abstand - keine Einrückung
\setlength\parskip{\medskipamount}
\setlength\parindent{0pt}

% Abstand Text - Graphik (nur Mitten in Text)
\setlength{\intextsep}{25pt plus 3pt minus 2pt}
%-----------------------------------------------------------------------
% EIGENE KOMMANDOS
\PScommands
% \todo{<text>}: TODO-Hinweis
\newcommand{\todo}[1]{\psboxit{box .7 setgray fill}{\spbox{TODO: [#1]}}\bigskip}
% \comment{<text>}: Kommentar, nicht im Dokument sichtbar
\newcommand{\comment}[1]{}
% \markup{<test>}: Unterstrichener Text
\newcommand{\markup}[1]{\uline{#1}}
% \file{<text>}: Formatierung für Dateinamen
\newcommand{\file}[1]{{\sffamily #1}}
% \code{<text>}: Formatierung von "Code" (Klassenname, Methodennamen etc.) im Fließtext
\newcommand{\code}[1]{\mbox{\texttt{#1}}}
% \begin{absolutelynopagebreaks}: prevent page breaks
%\newenvironment{absolutelynopagebreak}
%  {\par\nobreak\vfil\penalty0\vfilneg
%   \vtop\bgroup}
%  {\par\xdef\tpd{\the\prevdepth}\egroup
%   \prevdepth=\tpd}
\newenvironment{absolutelynopagebreak}
  {\par\nobreak\vfil\penalty0\vfilneg
   \vtop\bgroup}
  {\par\xdef\tpd{\the\prevdepth}\egroup
   \prevdepth=\tpd}
%-----------------------------------------------------------------------
% Spezielle TRENN-VORGABEN
\hyphenation{Tren-nung}
