\chapter{Anforderungsanalyse}

Das Hauptziel der Software ist die Bereitstellung von HomeMatic-Hausautomationsgeräten
im WieDAS-Datenraum.
Wie im Kapitel \ref{gru_aalcache} bereits aufgeführt, wird der WieDAS-Datenraum über eine Instanz
einer Cache-Software bereitgestellt.
Die HomeMatic-Geräte sind über eine Zentralkomponente mit dem Verarbeitungssystem verbunden.
Daher muss eine Zwischenkomponente (Gateway) entwickelt werden, welche bestimmten Anforderungen
gerecht wird.

Im Kapitel \ref{ana_ermittlung} wird untersucht, wie und wann die HomeMatic-Gerätedaten
abgerufen werden können.
Dabei wird die Schnittstelle der Geräte betrachtet und ermittelt welche Teile dieser Schnittstelle
für die Software relevant sind.

Der nächste Schritt ist die Bereitstellung der ermittelten Geräte in WieDAS.
Dazu müssen die Daten umgewandelt bzw. abgebildet werden, was im Kapitel \ref{ana_abb} näher erläutert wird.

Danach werden, wie in Kapitel \ref{ana_einpflegen} aufgezeigt wird, die umgewandelten Daten in den
WieDAS-Datenraum eingepflegt.
Dabei sind nicht nur Geräte zu berücksichtigen, welche zum Zeitpunkt der Programmausführung vorhanden sind,
sondern auch Geräte, die zur Laufzeit des Programms in der Umgebung hinzugefügt werden, wodurch weitere
Anforderungen entstehen.

In Kapitel \ref{ana_laufzeitumgebung} wird weiterhin überprüft auf welchen Geräten die Software eingesetzt wird
und welche Anforderungen daraus entstehen.
Daraus entscheidet sich auch die verwendete Programmiersprache der Implementierung.

Abschließend sind Konfigurationsanforderungen der Software in Kapitel \ref{ana_konfiguration} hinterlegt.

\section{Ermittlung von HomeMatic-Gerätedaten}
\label{ana_ermittlung}

\section{Abbildung der Gerätedaten}
\label{ana_abb}

\section{Einpflegen der Gerätedaten in WieDAS}
\label{ana_einpflegen}

\section{Laufzeitumgebung der Software}
\label{ana_laufzeitumgebung}

\section{Konfiguration des Systems}
\label{ana_konfiguration}
