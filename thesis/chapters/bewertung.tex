\chapter{Evaluation}
\label{bewertung}

In diesem Kapitel wird die Implementierung aus Kapitel \ref{implementierung} unter der
Berücksichtigung der Zielsetzung und der daraus entstandenen Anforderungen bewerten.
Zunächst wird aufgezeigt, wie viele Zeilen Code die einzelnen Module und andere Connectoren
aus dem Code-Repository \cite{aalcache_code} haben.
Dann werden die Anwendungsfälle aus Abschnitt \ref{ana_anforderungen} genutzt, um das
System zu validieren.
Abschließend wird erwähnt, welche nicht-funktionalen Anforderungen der Connector befriedigen
kann.

\section{Implementierungsaufwand}

\begin{table}[h]
\begin{tabular}{|l|l|}
\hline
Modul & Lines of Code\\
\hline
XML-RPC Methoden-Adapter & 142\\
HomeMatic-Typen für XML-RPC & 827\\
HomeMatic-Logik & 153\\
HomeMatic-Schnittstelle & 351\\
AAL-Cache-spezifische HomeMatic-Logik & 77\\
Abbildungs- und Verknüpfungsmodule für Geräte & 221\\
Geräteverwaltung & 330\\
Generator für Geräte und Verknüpfungen & 206\\
Templates für Geräte und Verknüpfungen & 135\\
\hline
& 2441\\
\hline
\end{tabular}
\caption{Anzahl der Codezeilen je Modul}
\label{tab_implauf}
\end{table}

Tabelle \ref{tab_implauf} zeigt die Anzahl der Codezeilen für jedes Modul.
Diese Zahlen wurden mit dem Programm \emph{cloc} \cite{cloc} ermittelt und beinhalten
weder Kommentare noch Leerzeilen.
Nicht aufgeführt ist der Implementierungsaufwand für Modultests, welche selbst in Modulen abgelegt sind.

\begin{table}[h]
\begin{tabular}{|l|l|}
\hline
Connector & Lines of Code\\
\hline
Plugwise-Connector & 729\\
DDS-Connector \cite{ddsconn} & 1136\\
Düsseldorf-Connector \cite{dconn} & 382\\
\hline
\end{tabular}
\caption{Anzahl der Codezeilen für andere Connectoren}
\label{tab_conns}
\end{table}

Tabelle \ref{tab_conns} zeigt die Anzahl der Codezeilen für verschiedene Connectoren aus dem
AAL-Cache Code-Repository \cite{aalcache_code}.
Dabei ist festzustellen, dass die Connectoren schlanker sind als die Implementierung des HomeMatic-Connectors.
Das liegt zum einen daran, dass die Protokolle häufig schlanker sind und dass die Connectoren nur die opaken
Cache-Einträge interpretieren müssen.
Ebenfalls werden für das korrekte Verarbeiten der zur Zeit verfügbaren Geräte nicht alle HomeMatic-Typen
für XML-RPC benötigt.
Weiterhin wird auch nicht die komplette HomeMatic-Schnittstelle benutzt.
Dennoch wurden beide Komponenten vollständig implementiert, so dass der volle Funktionsumfang der HomeMatic-Zentrale
im Connector nutzbar ist.

\begin{table}[h]
\begin{tabular}{|l|p{3cm}|l|}
\hline
Gerät & Lines of Code der Implementierung & Textzeilen der Modell-Datei\\
\hline
Dimmer & 206 & 146\\
Taster & 140 & 88\\
Dimmer-Verknüpfung & 135 & 101\\
\hline
\end{tabular}
\caption{Anzahl der Codezeilen für die Modell-Datei und die daraus entstehenden Implementierungen}
\label{tab_mods}
\end{table}
Anhand der Codezeilen für ein Gerät und dessen Modellierung aus Quellcode \ref{tab_mods} kann entnommen werden,
dass die Benutzung eines Generators zunächst nur wenig Einsparungen an Aufwand bedeutet.
Dies liegt daran, dass die Modellierung vergleichsweise viel Implementierung selbst schreiben muss.
Die Gemeinsamkeiten verschiedener Geräte beschränken sich auf wenige Details bei der Implementierung.

\section{Testszenarien anhand der Anwendungsfälle}
In diesem Abschnitt wird das System mit den Anwendungsfällen auf korrekte Funktionsweise überprüft.
Dabei wird sich mit dem Programm \emph{ssh} \cite{ssh} auf das System verbunden, auf dem der Connector
ausgeführt wird.
Dann wird der Connector gestartet und die Ausgaben beobachtet.
Ist der Connector in seinem Wartezustand, werden verschiedene Geräte an der Zentrale an- und abgemeldet.
Dazu wird die Zentrale eingeschaltet und an ein Netzwerk verbunden, so dass sie funktionsfähig wird.
Dann wird das WebUI der Zentrale in einem Browser aufgerufen und auf den rechtsstehenden \emph{Teach-In devices}-Button
geklickt.
Das Gerät welches für den Funktionstest verwendet wurde, war ein Funk-Handtaster HM-PB-2-WM55.
Dieses Gerät wird mit einem kurzen Drücken des Anmeldeknopfs auf der Hinterseite an der Zentrale angemeldet, während
sie im Anlernmodus ist.
Hat man den Knopf gedrückt, erscheint das Geräte im WebUI in der \emph{INBOX}.
Im WebUI müssen dann die Kanäle aufgelistet werden und der \emph{Transmission mode} auf \emph{Standard} gestellt werden,
da sonst keine Tastendrücke über die XML-RPC-Schnittstelle empfangen werden können.
In der Auflistung, welche über Klicks auf \emph{Settings} und \emph{Devices} erreicht wird der Taster rausgesucht
und auf \emph{Delete} geklickt.
Die Ausgaben des Connectors werden beobachtet und es wird geprüft, ob die Statusmeldungen bezüglich
der An- und Abmeldung erscheinen.

\lstset{language={}}
\begin{lstlisting}[frame=single,caption={Ausgaben des Connectors bei An- und Abmelden von HomeMatic-Geräten an der Zentrale},label=status]
HM-Connector: Added device HM-PB-2-WM55 (Address: KEQ0059462)
HM-Connector: Added device MAINTENANCE (Address: KEQ0059462:0)
HM-Connector: Added device KEY (Address: KEQ0059462:1)
HM-Connector: Added device KEY (Address: KEQ0059462:2)
[...]
DBG ccu_logic.cpp:102: HM-Connector: Removing device with address: KEQ0059462
HM-Connector: Removed all belongings of device with address: KEQ0059462
\end{lstlisting}

Quellcode \ref{status} zeigt die Ausgaben während des An- und Abmeldens des Tasters an der Zentrale.
Damit wäre der Anwendungsfall \ref{uc1} erfolgreich geprüft.

Der Anwendungsfall \ref{uc2} ist das Auslösen von Funktionen an HomeMatic-Geräten.
Danach soll der Connector die durchgeführte Funktion im WieDAS-Datenraum bekannt geben.
Dazu wurde ein Connector geschrieben, der die URI für den Taster für ein Callback registriert
und mitteilt, falls ein Ereignis eintritt.
Weiterhin muss eine Geräteabbildung modelliert werden (Abschnitt \ref{imp_abb_geräte}), die die
HomeMatic-Ereignisse in den WieDAS-Datenraum abbilden kann.
\lstset{language={}}
\begin{lstlisting}[frame=single,caption={Ausschnitt der Modellierung des Tasters},label=modtast]
[...]
hm_event: (event_data, channel_index)
	[...]
	if (event_data.value_key == "PRESS_SHORT") {
		auto const onofffunc_uri = std::string("/OnOffFunctionality") + temp_uri.str();
		int const tag = aalc_getTag(onofffunc_uri.c_str());
		WieDAS::OnOffFunctionality_dt functionality;

		functionality.id = get_wiedas_id();

		// even channel = on, uneven = off
		functionality.command = channel_index % 2 == 0 ? WieDAS::on : WieDAS::off;
\end{lstlisting}

Quellcode \ref{modtast} zeigt den verarbeitenden Teil (Zelie 2) der Modellierungsdatei eines Handtasters.
Dabei wird geprüft ob ein Tastendruck vorliegt (Zeile 4).
Falls dies der Fall ist, wird die URI gebildet (Zeile 5) und das Tag dazu angefordert (Zeile 6).
Dann wird ein Objekt der Funktionalität erstellt (Zeile 7) und die interne ID gesetzt (Zeile 9).
Es wird geprüft welche Seite des Tasters gedrückt wurde und dementsprechend der Ein- oder Auszustand
gesetzt (Zeile 12).
Nachdem das Gerät modelliert ist, wird es dem Generator (Abschnitt \ref{imp_abb_gen}) übergeben und
aus ihr eine dynamische Bibliothek erstellt.
Diese wird beim Programmstart geladen und stellt so dem Connector die Abbildung zur Verfügung.
Sobald der Connector auf Ereignisse wartet, wird eine Taste gedrückt.

\lstset{language={}}
\begin{lstlisting}[frame=single,caption={Ausgaben der Connectoren bei Ereignissen von HomeMatic},label=statushm]
HM-Connector: Event for device with address KEQ0059462:1 received
HM-Connector: Key: PRESS_SHORT
[...]
DBG watch_connector.cpp:13: Watch-Connector: Callback
DBG watch_connector.cpp:25: Watch-Connector: event for: /OnOffFunctionality/HMKEQ0059462-0
[...]
\end{lstlisting}

Quellcode \ref{statushm} zeigt die Ausgaben des Beobachtungs- und HomeMatic-Connectors während eine Gerätefunktion
ausgelöst wird.
Dabei wird zunächst das Ereignis wahrgenommen (Zeile 1 und 2) und dann intern die Verarbeitungsmethode
der Geräteabbildung aufgerufen.
Diese verteilt dieses Ereignis dann in den WieDAS-Datenraum (hier Ein/Aus-Ereignisse).
Der Watch-Connector der zuvor den entsprechenden URI registriert hat, meldet dass diese URI aktualisiert wurde (Zeile 4 und 5).
Da die Ereignisse vom Watch-Connector abgefangen werden können, sobald sie an einem HomeMatic-Gerät durchgeführt werden,
ist dieser Anwendungsfall ebenfalls erfolgreich durchführbar.

Für den Anwendungsfall \ref{uc3} wird neben der Geräteabbildung auch eine Modellierung der Geräteverknüpfung benötigt
(Abschnitt \ref{imp_verknüpfungen}).
Die Geräteverknüpfung die erstellt wurde, soll eine Verbindung zwischen einem Taster und einem Dimmer herstellen.
Der Taster soll den Dimmer ein- und ausschalten.
Den Aufbau der Modell-Datie der Geräteverknüpfung ist bereits in Quellcode \ref{conn_mod} aufgezeigt worden.
Beim Programmstart wird die dynamische Bibliothek, die der Generator erzeugt, geladen.
Der Connector lädt die Geräteabbildungen für den Dimmer und den Taster, als auch die Verknüpfungsbibliothek
und teilt kurz darauf mit, dass der Connector die URIs des Tasters registriert, da der URI einen
Sender identifiziert.
\lstset{language={}}
\begin{lstlisting}[frame=single,caption={Ausgaben des Connectors nach dem Laden von Verknüpfungsbibliotheken},label=statusvk]
HM-Connector: Adding /OnOffFunctionality/HMKEQ0059462-0 to subscription set because it's the sender of a connection.
HM-Connector: Adding /LightRegulationFunctionality/HMKEQ0059462-0 to subscription set because it's the sender of a connection.
[...]
\end{lstlisting}

Quellcode \ref{statusvk} zeigt einen Ausschnitt der Ausgaben nachdem die Bibliothek geladen wurde.
Wird nun ein Tastendruck ausgeführt, sorgt die Verknüpfung (Quellcode \ref{conn_mod}) dafür, dass der Dimmer
ein- bzw. ausgeschaltet wird.
\lstset{language={}}
\begin{lstlisting}[frame=single,caption={Ausgabe des Connectors bei Verarbeitung durch das Verknüpfungsmodul},label=statusonoff]
HM-Connector: Event for device with address KEQ0059462:1 received
HM-Connector: Key: PRESS_SHORT
DBG device_manager.cpp:309: HM-Connector: Subscripion handling for: /OnOffFunctionality/HMKEQ0059462-0
HM-Connector: Dimmer connection: state->1
DBG device_manager.cpp:340: HM-Connector: Subscription for /OnOffFunctionality/HMKEQ0059462-0 handled.
[...]
HM-Connector: Event for device with address JEQ0193620:1 received
HM-Connector: Key: LEVEL
DBG device_manager.cpp:309: HM-Connector: Subscripion handling for: /OnOffOutput/HMJEQ0193620-0
DBG device_manager.cpp:340: HM-Connector: Subscription for /OnOffOutput/HMJEQ0193620-0 handled.
\end{lstlisting}

Quellcode \ref{statusonoff} zeigt die Ausgabe des Connectors bei einem Tastendruck.
In Zeile 1 und 2 wird das Ereignis des Tastendrucks wahrgenommen und danach verarbeitet (siehe Test des vorherigen Anwendungsfall).
Dann wird die Callback-Funktion aufgerufen, da das Ein/Aus-Ereignis im WieDAS-Datenraum erscheint (Zeile 3).
Die Verarbeitungsmethode des Verknüpfungsmoduls wird aufgerufen, welche dann das Dimmer-Ereignis für den WieDAS-Datenraum
erstellt und dem AAL-Cache mitteilt (Zeile 4).
Schließlich wird auch dann wieder das WieDAS-Ereignis des Dimmers vom Connector registriert (Zeile 7) und die Ausgangsleistung
entsprechend erhöht oder verringert (Zeile 7-9).
Durch das Einstecken einer Lampe in den Dimmer, kann die Funktion auch außerhalb des Verarbeitungssystems sichtlich überprüft werden.
Der Anwendungsfall ist also korrekt durchführbar.

\section{Befriedigte nicht-funktionale Anforderungen}
Während der Design- und Implementierungsphase wurden die nicht-funktionalen Anforderungen aus Abschnitt \ref{ana_anforderungen} 
stets versucht berücksichtigt zu werden.
Um einen Überblick des Speicherverbrauchs zur Laufzeit zu erlangen, wurde das Programm \emph{ps} \cite{procps} auf dem
ausführenden System ausgeführt.

\begin{table}[h]
\begin{tabular}{|l|l|}
\hline
\multicolumn{2}{|c|}{Speicherverbrauch in Kilobytes}\\
\hline
ohne Connector & mit Connector\\
\hline
944 & 5612\\
\hline
\end{tabular}
\caption{physikalischer Speicherverbrauch von AAL-Cache in Kilobytes}
\label{tab_mem}
\end{table}

Tabelle \ref{tab_mem} zeigt den RAM-Speicherverbrauch des Programms zunächst
vor dem Laden des Connectors und danach.
Dabei fällt auf, dass der Speicherverbrauch stark steigt.
Bei genauerer Inspektion mit dem Programm \emph{pmap} \cite{procps} fiel auf,
dass nur vergleichsweise wenig Speicherverbrauch vom Connector und den Geräten selbst benötigt wird
(ca. 1200KB).
Ein Großteil des Speicherverbrauchs liegt in den Bibliotheken und deren Abhängigkeiten.
Die XML-RPC-Bibliothek benötigt die SSL- und diverse Cryptobibliotheken des Systems, obwohl
die SSL-Funktionalität gar nicht benötigt wird.
Ebenfalls hängt der verbrauchte Speicher von den Compileroptionen ab, wie die nächste
Tabelle zeigt.

\begin{table}[h]
\begin{tabular}{|l|l|}
\hline
Datei & Größe in Kilobytes\\
\hline
\multicolumn{2}{|c|}{mit Debuggingsymbolen}\\
\hline
aalcache & 1333\\
libhomematic\_conn.so & 3103\\
dyn-devices/libtrailing\_edge\_dimmer.so & 509\\
dyn-devices/libhandtransmitter.so & 465\\
dyn-connections/dimmer.so & 320\\
\hline
\multicolumn{2}{|c|}{mit Optimierungen des Compilers}\\
\hline
aalcache & 61\\
libhomematic\_conn.so & 252\\
dyn-devices/libtrailing\_edge\_dimmer.so & 49\\
dyn-devices/libhandtransmitter.so & 36\\
dyn-connections/dimmer.so & 25\\
\hline
\end{tabular}
\caption{Größe der relevanten Dateien in Kilobytes}
\label{tab_sizes}
\end{table}

Die Tabelle \ref{tab_sizes} zeigt wie unterschiedlich die Größen der Bibliotheken und des Caches
sein können, wenn dem Compiler Optimisierungen erlaubt werden und die Debuggingsymbole weggenommen werden.
Für die Tests sind die GCC-Compileroptionen \emph{-O2}, \emph{-DNDEBUG} und \emph{-ggdb3} entsprechend
hinzugefügt oder entfernt worden.
Durch die Verwendung der Bibliotheken ist die Anforderung der geringen Speichernutzung nur
innerhalb des Connectors erfüllt.
Der gesamte Speicherverbrauch für den geladenen Connector im AAL-Cache scheint zu groß.

Die durchschnittliche Anzahl von logischen Lines of Code in einer Funktion beträgt 7,84.
Dieses Ergebnis wurde durch eine Messung des Programms \emph{Loc-Metrics} \cite{locmetrics}
herausgefunden.
Damit ist die Größe der Sub-Routinen im Durchschnitt sehr gering und die nicht-funktionale Anforderung
bezüglich kleinen und leicht-optimierbaren Subroutinen erfüllt.

Komplexere Datenstrukturen gibt es im Connector nicht.
Die Datenstrukturen, die von der C++-Standardbibliothek genutzt werden und Komplexität besitzen sind
Listen, Assoziationen und referenzzählende Zeigercontainer.

Die Energieeffizienz des laufenden Systems wurde nicht bemessen.
Während der Connector auf Ereignisse wartet ist die Kontrolle an die XML-RPC-Bibliothek übergeben.
Diese wartet mit einem Polling auf dem Betriebssystem-Socket auf eintretende Ereignisse.
Die Vorgehensweise ist üblich und für den Betrieb auf eingebetteten Geräten kein Hindernis.
Der Connector selbst verwendet kein Polling.
Er benutzt für die Verhinderung des gemeinsamen Zugriffs auf Datenobjekte ein Locking-Mechanismus.
Dieser ist nicht fein-granular, durch die Codestelle des Lockings ist dies aber nicht weiter schlimm.
Das Locking wird nur verwendet, wenn ein Ereignis aus dem WieDAS-Datenraum oder von HomeMatic eintritt, so
dass beide Ereignisse unabhängig voneinander komplett verarbeitet werden können ohne das ein gemeinsamer
Datenzugriff entstehen kann.

