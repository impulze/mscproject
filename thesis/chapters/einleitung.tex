\begin{abstract}

In der Thesis wird eine Software entwickelt, welche es erlaubt HomeMatic-Hausautomationsgeräte
in die, von der Hochschule RheinMain und Fachhochschule Düsseldorf, im Rahmen von Forschungsprojekten
entwickelte AAL-Dienstplattform WieDAS zu integrieren.
Die Software bildet eine Adapterkomponente zwischen der von HomeMatic-Geräten angebotene Schnittstelle
und der WieDAS Komponente für Datenerfassung und Datenerhaltung.
Darin enthalten ist eine Abbildung der Geräte in die WieDAS-Sprache zur Beschreibung von Geräten.
Ziel ist es, die HomeMatic-Geräte, welche von WieDAS unterstützt werden können, vollständig in die
Dienstplattform einzubinden.
Die Thesis vermittelt zunächst Grundlagen der verwendeten Systeme und zeigt die Anforderung an eine
solche Softwarekomponente.
Darauf folgen Überlegungen zur Konzipierung und der Darstellung des gewählten Designs.
Das implementierte Design wird erklärt und anschließend mit einigen Laufzeittests bezüglich
der Performance und dem Speicherverbrauch getestet.
