\chapter{Einleitung}

Da in der heutigen Zeit der prozentuale Anteil der einsamen und älteren Bevölkerung steigt,
versucht man diesen ``demografischen Wandel'' mit der Nutzung und Verbreitung technischer
Hilfsmittel im Alltag entgegenzuwirken.
Der Begriff des \emph{Ambient Assisted Living} umfasst die Produkte und Dienstleistungen, welche
diese Hilfsmittel ermöglichen.

In der Thesis wird eine Software entwickelt, welche es erlaubt Hausautomationsgeräte der Produktreihe
\emph{HomeMatic} der Firma \emph{eQ-3} in die, von der Hochschule RheinMain und Fachhochschule Düsseldorf,
im Rahmen von Forschungsprojekten entwickelte, AAL-Dienstplattform WieDAS zu integrieren.
Die Dienstplattform erfasst und verarbeitet Daten die von verschiedenen Geräten und somit über
verschiedene Protokolle an den WieDAS-Datenraum angebunden sind.

Die Software bildet eine Adapterkomponente zwischen der von HomeMatic-Geräten angebotene Schnittstelle
und der WieDAS-Schnittstelle für Datenerfassung und Datenerhaltung.
Darin enthalten ist eine Abbildung der Geräte in die WieDAS-Beschreibungssprache für Geräte.
Ziel ist es, die HomeMatic-Geräte, welche von WieDAS unterstützt werden können, vollständig in die
Dienstplattform einzubinden, so dass Geräte herstellerunabhängig miteinander verbunden werden können.

Die Thesis vermittelt zunächst Grundlagen der verwendeten Systeme in Kapitel \ref{grundlagen}.
Folgend werden die Problemfelder, die bei der Entwicklung eines solchen Adapters auftreten, in
Kapitel \ref{analyse} analysiert.
In Kapitel \ref{ana_uc} werden typische Anwendungsfälle aufgelistet und näher betrachtet.

Darauf folgen Überlegungen zur Konzipierung und der Darstellung des gewählten Designs 
in Kapitel \ref{design} unter der Berücksichtigug der Designentscheidungen aus der Analyse.
Das implementierte Design wird in Kapitel \ref{implementierung} erklärt und anschließend mit
einigen Laufzeittests bezüglich der Performance und dem Speicherverbrauch getestet.
Neben den Tests findet in Kapitel \ref{bewertung} auch eine Bewertung der Software bezüglich
Aufwand der Konzipierung und Implementierung und Verwendbarkeit (insbesondere statische
Modularität, Kompatibilität und Möglichkeiten für zukünftige Erweiterungen).

Das Kapitel \ref{fazit} beendet die Thesis fasst die erreichten Ziele zusammen und zeigt
offene Probleme und Softwarekomponenten, die nicht implementiert werden konnten.
Weiterhin wird gezeigt, wie die Software weiterentwickelt werden kann, um diese Probleme
zu beheben und Implementationslücken zu füllen.
