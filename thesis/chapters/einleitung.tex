\chapter{Einleitung}

Immer mehr Haushalte benutzen Automatisierungstechnik oder Geräte mit Bezug zur Informationstechnik.
Diese dient dazu, in Haushalten mit dem Einsatz von moderner Technik und Technologien einen gewissen Komfort
zu ermöglichen.
So werden z.B. bei der Elektroinstallation elektronische Jalousiemotoren verbaut, so dass diese
um eine bestimmte Uhrzeit automatisch rauf- oder runter gefahren werden können.
Auch findet man in modernen Installationen oft intelligente Regelungen von Heizungssystemen.
Neuere Entwicklungen beziehen bei der Verknüpfung und Verwendung von Geräten weiterhin die
Unterhaltungselektronik mit ein.
So ist es z.B. möglich, an jedem Standort im Haus auf einen zentralen Medienserver über Netzwerktechnologie
zuzugreifen und Audio- und Videomedien abzurufen.
Diesen Aspekt der Vernetzung zusammen mit der Möglichkeit, diese vernetzten Komponenten über eine
bereitgestellte Logikschnittstelle anzusprechen nennt man Smart Home.

Gängige Übertragunstechnologien, die in diesem Umfeld zum Einsatz kommen, sind z.B.
Zigbee Pro \cite{zigbee_p}, \emph{HomeMatic-BidCos ®} \cite{homematic_eq3} und andere WLAN basierte
Technologien.
Diese unterscheiden sich in mehreren Kategorien z.B. Länge des Übertragungsweg, Energieeffizienz und
Sicherheitsmerkmale.

In diesem Kontext der Hausautomatisierung findet man auch das \emph{Ambient Assisted Living} (AAL).
Der prozentuale Anteil der alleinstehenden und älteren Bevölkerung steigt stetig.
Dies nennt man den ``demografischen Wandel''.
Durch die Nutzung und Verbreitung technischer Hilfsmittel im Alltag versucht man, diesen Menschen
zu helfen.
Der Begriff des \emph{Ambient Assisted Living} umfasst die Produkte und Dienstleistungen, welche
diese Hilfsmittel ermöglichen.

AAL ist in der Hinsicht auf Interoperabilität noch nicht weit ausgereift, da es noch keine weit verbreiteten
Techniken und Technologien bzw. die dazugehörigen Standards existieren.
Um die Entwicklungsarbeit von AAL-Plattformen zu erleichtern, existieren Frameworks wie OSGi\cite{osgi}.
Diese bieten, durch Bereitstellen von Schnittstellen, eine einheitliche Sicht auf verschiedene Dienste,
die vom Entwickler implementiert werden müssen (z.B. Logging, Gerätezugriff und Remote-Konfiguration).
Das OSGi dient gleichzeitig als Ausgangspunkt für die AAL-Dienstplattform WieDAS \cite{wiedas}, welche
im Rahmen eines Forschungsprojekts der Hochschule RheinMain und der Fachhochschule Düsseldorf entstand.
Diese Plattform wurde unter dem Ansatz entwickelt, AAL mit dem Smart-Home-Kontext erfolgreich
zu verknüpfen.
WieDAS besitzt eine Beschreibungssprache zum Modellieren von Hausautomationsgeräten.
Der Datenraum, in dem sich die Geräte befinden, wird über eine Programminstanz angeboten, welches
eine Schnittstelle besitzt, um die Geräte zu identifizieren und über Ereignisse von Ihnen
benachrichtigt zu werden.

Im Rahmen der Thesis wird eine Software entwickelt, welche es erlaubt, Hausautomationsgeräte der Produktreihe
\emph{HomeMatic} der Firma \emph{eQ-3} in die WieDAS zu integrieren.
WieDAS dient der Vereinheitlichung von unterschiedlichen Daten, die von verschiedenen Geräten ähnlichen
Typs stammen.
Diese Geräte kommunizieren über proprietäre Protokolle miteinander und mit der Zentrale.
Die Zentrale selbst wiederum bietet eine Schnittstelle, die über das Netzwerk benutzt werden kann, um die
angeschlossenen Geräte zu verwalten und konfigurieren.

Die zu entwickelnde Software bildet eine Adapterkomponente zwischen der von HomeMatic- und Datenraum-Schnittstelle von
WieDAS.
Darin enthalten ist eine Abbildung der HomeMatic-Geräte in die WieDAS-Beschreibungssprache.
Ziel ist es, die HomeMatic-Geräte, welche von WieDAS unterstützt werden können, vollständig in die
Dienstplattform einzubinden, so dass Geräte herstellerunabhängig miteinander verbunden werden können.

Die Thesis vermittelt zunächst Grundlagen der verwendeten Systeme in Kapitel \ref{grundlagen}.
Folgend werden die Problemfelder, die bei der Entwicklung eines solchen Adapters auftreten, in
Kapitel \ref{analyse} analysiert.
In Abschnitt \ref{ana_uc} werden typische Anwendungsfälle aufgelistet und näher betrachtet.

Darauf folgt die Darstellung des Konzepts in Kapitel \ref{design} unter der Berücksichtigug der
Designentscheidungen aus der Analyse.
Das implementierte Design wird in Kapitel \ref{implementierung} erklärt.
Neben den Tests bezüglich der Performance und des Speicherverbrauchst befindet sich in Kapitel
\ref{bewertung} eine Bewertung der Software bezüglich der Implementierung und Verwendbarkeit
(z.B. Modularität, Kompatibilität und Möglichkeiten für zukünftige Erweiterungen).

Das Kapitel \ref{fazit} fasst die erreichten Ziele zusammen und zeigt
offene Probleme und nicht implementierte Funktionalität.
Weiterhin wird gezeigt, wie die Software weiterentwickelt werden kann, um diese Probleme
zu beheben und Implementationslücken zu füllen.
