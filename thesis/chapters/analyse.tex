%Diese zentrale Station kann darüber informiert werden, dass neue Geräte in der Umgebung hinzugefügt
%oder aus ihr entfernt wurden.

\chapter{Analyse}
\label{analyse}

\section{Anforderungen}
\label{ana_anforderungen}

Das Hauptziel der Software ist die Bereitstellung und Verwendung von HomeMatic-Hausautomationsgeräten
im WieDAS-Datenraum.
Wie im Kapitel \ref{gru_aalcache} bereits aufgeführt, wird der WieDAS-Datenraum über eine Instanz
einer Cache-Software bereitgestellt.
Die HomeMatic-Geräte sind über eine Zentralkomponente mit dem Verarbeitungssystem verbunden.
Es wird ein Connector \ref{gru_aalcache} entwickelt, der die HomeMatic-Geräte an den Cache koppelt
und bestimmten Anforderungen gerecht werden muss.
In der Analyse werden häufiger Lösungsansätze vorgeschlagen oder ihre Vor- und Nachteile aufgezeigt.

Die Anforderungen werden zunächst durch die Bereitstellen von Anwendungsfällen in Kapitel \ref{ana_uc}
veranschaulicht.

Der erste Analyseschritt ist dann die Betrachtung der Herangehensweise bei der Entwicklung der
Software in Kapitel \ref{ana_sw}.

Die erste Aufgabe des Connectors, ist die Bereitstellung der ermittelten Geräte in WieDAS.
Dazu wird zunächst das Gerätemodell von WieDAS in Kapitel \ref{ana_wd_modell} analysiert.
Durch die Betrachtung des Modells kann ermittelt werden, welche Informationen aus den HomeMatic-Geräten
benötigt werden.

Das Kapitel \ref{ana_hm} untersucht, wie HomeMatic-Geräte beschrieben sind und wie diese
Beschreibungen abgerufen werden können.
Dabei wird auch die Schnittstelle betrachtet, die für das Anfordern der Gerätedaten relevant ist und
wie die übertragenen Informationen interpretiert werden müssen.

Da die HomeMatic-Geräte in WieDAS eingepflegt werden, müssen eine Abbildung des Modells erstellt werden.
Dazu müssen die Strukturen, Datentypen und evtl. Schnittstellen umgewandelt bzw. abgebildet werden,
was im Kapitel \ref{ana_abb} näher erläutert wird.

Die Arbeitsweise des Connectors wird in Kapitel \ref{ana_connector} geschildert.
Der Connector ist nicht nur für das initiale Befüllen des WieDAS-Datenraums mit HomeMatic-Geräten
zum Programmstart verantwortlich, sondern reagiert auch während der Laufzeit auf neue und entferne Geräte
und stellt Signale an verschiedene Geräte zu.

Kapitel \ref{ana_laufzeitumgebung} wird weiterhin überprüft auf welchen Geräten die Software eingesetzt wird
und welche Anforderungen daraus entstehen.
Daraus entscheidet sich auch die verwendete Programmiersprache der Implementierung.

Abschließend sind Konfigurationsanforderungen der Software in Kapitel \ref{ana_konfiguration} hinterlegt.

Die Entscheidungen die getroffen wurden, und dem Design der Software zu Grunde liegen, werden in
Kapitel \ref{ana_entscheidungen} dokumentiert.

\subsection{Entwicklung}

Connectoren werden in der Umgebung der Cache-Software entwickelt.
In diesem Kapitel wird untersucht, welche Anforderungen an die Entwicklung gestellt werden.

\subsection{WieDAS-Gerätemodell}

Das WieDAS-Gerätemodell wird durch eine IDL beschrieben.
Es wird untersucht, welche Struktur ein Gerät aufweist und welche Datentypen oder Schnittstellen
zur Verfügung stehen.

\subsection{HomeMatic-Geräte}

Die Geräte der HomeMatic-Hausautomation kommunizieren mit der HomeMatic-CCU.
Weiterhin ist sie für Konfiguration und Abrufen der Gerätedaten zuständig (Kapitel \ref{gru_hm_ccu}).
Es muss analysiert werden, wie das XML-RPC Protokoll, welches die HomeMatic-CCU anbietet,
angewendet wird, um Gerätedaten abzurufen und wie diese aufgebaut sind.

\subsection{Abbildung der Gerätedaten}

Die Daten der Hausautomationsgeräte unterliegen einer gewissen Struktur, welche aus verschiedenen
Datentypen zusammengesetzt ist und in Kapitel \ref{ana_hm_modell} analysiert wurden.
Auch in WieDAS ist die Struktur einer Entität aus verschiedenen Datentypen zusammengesetzt (Kapitel \ref{ana_wd_modell}).
Diese Strukturen müssen dahingehend untersucht werden, dass eine nahezu vollständige Abbildung
zwischen den beiden Strukturen ermöglicht wird.

\subsection{Aufbau und Arbeitsweise des Connectors}

Der Kern-Anwendungsfall der Arbeit ist die Steuerung und das Reagieren auf Befehle in einer
heterogenen Hausautomationsumgebung.
Dieser Anwendungsfall hat Anforderungen an den Connector, da man bestimmte Mechanismen zur
Benachrichtigung von Geräten nutzen oder implementieren muss.

\subsection{Laufzeitumgebung der Software}

Das Gateway, welches die HomeMatic-Geräte anbindet muss in einer für Hausautomation typischen
Umgebung (z.B. Netzwerken mit Router oder andere Embedded-PC) lauffähig sein.
Dadurch entstehen weitere Anforderungen bzgl. der Implementierungssprache, der Einschränkungen
von Speichernutzung, verwendete Bibliotheken und ähnliches.

\subsection{Konfiguration des Systems}

Da das Gateway auf verschiedenen Geräten lauffähig sein soll und in einem beliebigen Netzwerk
agiert, müssen bestimmte Eigenschaften des Systems statisch und ggf. dynamisch konfiguriert werden.
Dafür muss ein Zugang zur Konfigurationseinheit bereitgestellt werden.

\section{Anwendungsfälle}
\label{ana_uc}

Für die Anwendungsfälle wird davon ausgegangen, dass sich das System in einem lauffähigen Zustand
befindet.
Das bedeutet, dass
\begin{itemize}
\item die notwendigen Netzwerkeinstellungen vorgenommen hat
\item der Connector in der Konfigurationsdatei korrekt eingetragen wurde
\item die Konfiguration für den Connector den Netzwerkeinstellungen entspricht
\end{itemize}
Die Anwendungsfälle \emph{UC02} bis \emph{UC04} setzen einen direkten Zugriff
auf die Laufzeitumgebung des AAL-Cache voraus (z.B. mit der Fernsteuerungssoftware \emph{SSH}).

\begin{itemize}
\item[UC01] Starten des AAL-Cache mit eingebundenem Connector
\item[Standardablauf]
 \begin{enumerate}
 \item Der AAL-Cache wird auf der Kommandozeil gestartet
 \end{enumerate}
\item[Ergebnis]
 \begin{enumerate}
 \item Der AAL-Cache meldet wieviele Geräte gefunden worden sind und ob Fehler aufgetreten sind
 \end{enumerate}
\end{itemize}

\begin{itemize}
\item[UC02] Anmelde eines neuen HomeMatic-Geräts
\item[Standardablauf]
 \begin{enumerate}
 \item Das Gerät wird laut Gerätebeschreibung in der HomeMatic-CCU angemeldet
 \end{enumerate}
\item[Ergebnis]
 \begin{enumerate}
 \item Das Gerät taucht in der HomeMatic-CCU auf. 
 \item Das Anmelden beim Connector wird mit einer kurzen Statusmeldung bestätigt.
 \end{enumerate}
\end{itemize}

\begin{itemize}
\item[UC03] Löschen eines Geräts
\item[Standardablauf]
 \begin{enumerate}
 \item Ein angemeldetes Gerät (siehe \emph{UC02}) wird im WebUI abgemeldet
 \end{enumerate}
\item[Ergebnis]
 \begin{enumerate}
 \item Das Gerät erscheint nicht mehr in der HomeMatic-CCU.
 \item Das Abmelden beim Connector wird mit einer kurzen Statusmeldung bestätigt.
 \end{enumerate}
\end{itemize}

\begin{itemize}
\item[UC04] Durchführen eines Schaltbefehls
 \begin{enumerate}
 \item Ein angemeldetes Schaltgerät führt verschiedene Schaltbewegungen aus (Lang, Kurz, Oben, Unten)
 \end{enumerate}
\item[Ergebnis]
 \begin{enumerate}
 \item Der Connector meldet einen erfolgreichen Schaltvorgang
 \end{enumerate}
\end{itemize}

\section{Entwicklung}
\label{ana_sw}

Connectoren sind sehr eng an den AAL-Cache gekoppelt und von der Umgebung der eigentliche Geräte
entkoppelt (siehe Kapitel \ref{gru_wiedas_plattform}).
Da es schon verschiedene Connectoren gibt, kann man sich vorhandene Implementierungen und die
verwendeten Werkzeuge, die bei der Entstehung verwendet wurden, betrachten.
Das Code-Repository des AAL-Caches enthält die Schnittstellen und deren Beschreibungen, sowie
Bauanleitungen und verschiedene Connectoren-Implementierungen.

Die Connectoren, die im Repository enthalten sind, werden mit ANT-Dateien und Makefiles erstellt.
Das Ergebnis ist eine dynamische Bibliothek, wessen Aufenthaltsort dem Connector über eine
Konfigurationsdatei (connectors.conf) mitgeteilt wird.
Die verwendete Toolchain, sowie Bauanleitungen müssen also in der Lage sein dynamische Bibliotheken
zu erstellen.

Die WieDAS-Geräte werden mit der Beschreibungssprache IDL beschrieben.
Eine IDL ist ähnlich aufgebaut wie eine Schnittstellenbeschreibung in der Programmiersprache C.
Es existieren bereits Konvertierungsmöglichkeiten von IDL zu C und C++, wie z.B. von der
DCE-Spezifikation \cite{cde} oder von Orbix \cite{orbix}.
Für weitere Konvertierung zu anderen Sprachen existieren Code-Konvertierungsprogramme wie SWIG \cite{swig}.

Falls der Aufwand einer eigenständigen Konvertierung in die verwendete Zielsprache zu hoch ist, sollte
auf solche automatischen Konvertierungen zurückgegriffen werden.
Die Entscheidung sollte von der Implementierungssprache und dem Umfang der IDL abhängig gemacht werden.
Da die IDL zu diesem Zeitpunkt noch nicht vollständig ist \cite{wiedas_onto},
könnte eine zukünftige Verwendung von Generatoren im Entwicklungsprozess vorgesehen werden.

\section{WieDAS-Gerätemodell}
\label{ana_wd_modell}

WieDAS selbst wird durch eine Instanzierung der AAL-Cache Software (Kapitel \ref{gru_aalcache})
repräsentiert.
Diese Softwarekomponente dient der Konfiguration des WieDAS-Datenraums, sowie dem Auslesen und
Hinzufügen von beteiligten Geräten in der Umgebung.
Diese Geräte werden in einer IDL \cite{idl} spezifiziert \cite{wiedas_idl}.
Es existieren wenige primitive Datentypen, darunter z.B. eine ID, Schaltzustände oder Aufzählungen.
Weiterhin werden gewöhnliche Aufgaben, wie das Erfassen einer Temperatur, das Schalten, oder ähnliches
in Funktionalitäten zusammengefasst.
Die Geräte besitzen kaum spezifische Daten, dafür jedoch aussagekräftige kleinere Daten.
Das liegt zum Einen daran, dass WieDAS versucht Geräte verschiedener Hersteller einzubinden und zum
anderen daran, dass sich der Speicherverbrauch der Geräteinformationen im AAL-Cache deutlich steigt,
wenn einfache Gerätefunktionen mit mehreren Daten abgebildet werden.

Einzelne speziellere Geräte besitzen eine eigene Struktur, welche wiederum genauso aufgebaut sein
kann, wie eine Funktionalität oder andere Geräte.
Zum Beispiel ist die Struktur eines Wasserstandsmelders identisch mit der Datenstruktur einer
normalen Lampe, da sie beide nur Informationen bezüglich des Einschaltzustands haben (An oder Aus).
Sind diese Geräte komplexer, wie z.B. eine dimmbare Lampe oder eine einstellbare Steckdose, so werden
weitere Parameter dieser Geräte mit vorher spezifizierten primitiven Datentypen oder den
Zahlentypen aus der IDL-Spezifikation erweitert.

\begin{absolutelynopagebreak}
Ein typisches Gerät besteht aus einer ID und einer für die Funktion relevanten Datentyp.

\lstset{language=IDL}
\begin{lstlisting}[frame=single,caption={Gerätebeschreibung eines dimmbaren Lichts in WieDAS}]
typedef unsigned short deviceID_t;
enum OnOff { off, on };
typedef OnOff OnOffState_t;
struct DimmableLight_dt {
	deviceID_t id;
	octet LightIntensityState;
	OnOffState_t state;
};
\end{lstlisting}
\end{absolutelynopagebreak}

Die WieDAS-IDL muss für die verwendete Implementierungssprache so interpretiert werden, dass die
sprachabhängigen Typen die Typeigenschaften (insbesondere Restriktionen) der IDL (z.B. Wertebereich)
unterstützen bzw. abdecken.
Daher ist eine Konvertierung in die verwendete Implementierungssprache nötig (siehe Kapitel \ref{ana_sw}).

\section{HomeMatic-Geräte}
\label{ana_hm}

\subsection{Gerätemodell}
\label{ana_hm_modell}
Die Geräte werden aus einer Menge von Kanälen modelliert \cite[Seite 13]{hmscript2} in welchen
die eigentliche Funktionalität steckt \cite[Seite 16]{hmscript2}.
Diese Funktionalität wird in Datenpunkten abgebildet.
Jedes Gerät besitzt eine verschiedene Anzahl und Art von Kanälen mit verschiedenen Datenpunkten.
Die Datenpunkte werden dann entweder ausgelesen um Zustände von Geräten zu erfahren oder gesetzt
um Funktionalitäten zu erfüllen.
Ein Gerät wird über eine Adresse identifiziert.

Der Schnittstellenprozess der CCU behandelt nur logische Geräte \cite{homematic_xmlrpc}.
Somit werden auch Funktionen der Geräte als eigenständiges Gerät bezeichnet und
erhalten eine eigenständige Adresse.
Ein Multifunktionsgerät könnte folgendermaßen adressiert werden:
\begin{itemize}
\item ABC123 Gerät selbst
\item ABC123:1 Schaltausgang 1
\item ABC213:3 Schalteingang 3
\end{itemize}

Die Datenpunkte sind typisiert, haben bestimmte Eigenschaften und Metadaten \cite[Seite 21]{hmscript2}.
Die Eigenschaften und Metadaten eines Datenpunkts sind der aktuelle typisierte Wert, den letzten Wert
(z.B. sinnvoll für dimmbare Aktoren), die Operationen die auf dem Datenpunkt ausgeführt werden können
und den Zeitstempel der letzten Aktualisierung.

Die Kanäle und die dazugehörigen Datenpunkte der zur Zeit verfügbaren Geräte sind
spezifiziert \cite{hmscript4}.
Die folgenden Tabellen zeigen Ausschnitte aus dem Modell für ein
HomeMatic-Funk-Wandthermostat \cite[Seite 12]{hmscript4}.

\begin{table}[h]
\begin{tabular}{|l|l|l|}
\hline
Kanaltyp & Kanalnummer \\
\hline
WEATHER & 1 \\
\hline
CLIMATECONTROL\_REGULATOR & 2 \\
\hline
\end{tabular}
\caption{Kanaltypen eines HomeMatic-Wandthermostats}
\label{tab_hm_chan}
\end{table}

\begin{table}[h]
\begin{tabular}{|l|l|l|l|}
\hline
Kanal & Name & Typ & Zugriff \\
\hline
1 & TEMPERATURE & float & lesend und über Ereignisse \\
\hline
2 & SETPOINT & boolean & lesend, schreibend und über Ereignisse \\
\hline
\end{tabular}
\caption{Datenpunkte eines HomeMatic-Wandthermostats}
\label{tab_hm_dp}
\end{table}

Die Tabelle \ref{tab_hm_chan} zeigt, dass das Thermostat 2 Kanäle besitzt.
Ein Kanal existiert für die Zustandsinformationen bezüglich der Wetterdaten, ein anderer
für das Interagieren mit der Regulierung.
Die Datenpunkte in Tabelle \ref{tab_hm_dp} sind beide direkt auszulesen oder können
Ereignisse auslösen (z.B. wenn sich die Raumtemperatur ändert oder die Stelltemperatur
gesetzt wird).
Nicht aufgezeigt sind in dieser Tabelle weiterführende Attribute der Datenpunkte, welche
den Wert genauer beschreiben \cite[Seite 3]{hmscript4}.

\begin{itemize}
\item Minimalwert
\item Maximalwert
\item Standardwert
\item Einheit
\item Spezielle Werte
\end{itemize}

Diese Wertebeschreibung ist nur dort sinnvoll, wo der Datentyp numerisch ist.
Spezielle Werte treten auf, um bestimmte Zustände zu beschreiben.
Zum Beispiel gibt es für den Datenpunkt \emph{SETPOINT} in Tabelle \ref{tab_hm_dp}
die speziellen Werte \emph{VENT\_CLOSED} und \emph{VENT\_OPEN} um den Zustand des angeschlossenen
bzw. gekoppelten Ventils zu beschreiben.
Diese Werte sind speziell zu interpretieren und liegen außerhalb des Wertebereichs
(hier 0 und 100).

Datenpunkte werden mit ihrem Namen und dem dazugehörigen Wert im XML-RPC als Parametersets
zusammengefasst .
Diese Parametersets haben Beschreibungen, die ebenfalls mit Funktionen vom Schnittstellenprozess
angefordert werden können.
Parametersets sind dann wiederum noch einmal nach Typ kategorisiert.
\begin{enumerate}
\item MASTER dient der Konfiguration eines Geräts
\item LINK offeriert ein Parameterset zur Beschreibung einer Verknüpfung zwischen Kanälen
\item VALUES ist das Set, welches die Messwerte, Statusinformationen und Schaltzustände beschreibt
\end{enumerate}

Die Aufgabe des Connectors ist es, diese Datenpunkte anhand Ihrer Namensgebung zu interpretieren.
Eine allgemeine Dokumentation darüber, wie die Datenpunkte zu interpretieren sind, liegt leider
nicht vor.

Für die Implementierung werden die Geräte als Datenstruktur benötigt.
Falls dafür ein Generator geschrieben werden soll (siehe Kapitel \ref{ana_sw}), so müssen
die Geräte zunächst mit einer Beschreibungssprache vorliegen, da sie bisher nur in
PDF Dokumenten von HomeMatic dokumentiert sind.

\subsection{XML-RPC Schnittstelle}
\label{ana_hm_xmlrpc}

Für das Abrufen von Gerätedaten aus der zentralen Steuereinheit (HomeMatic-CCU) wird das XML-RPC
Protokoll verwendet.
Das Protokoll verwendet HTTP für den Transport der Daten (Kapitel \ref{gru_xmlrpc}).
Dadurch muss die Software einen HTTP-Client implementieren, der in der Lage ist die geforderte
Datenstruktur des Protokolls an die HomeMatic-CCU zu übertragen.
Da die Anforderung von Gerätedaten nicht häufig vorkommt, muss die Implementierung keinen großen
Wert auf die Performance des Clients legen.
Die XML-Struktur ist durch das Protokoll vordefiniert und kann daher in statischer Form als
Template vorliegen und gefüllt werden oder mit parametrisierten Subroutinen erstellt werden.

Die HomeMatic-CCU bietet Methoden an, die für die Erfassung von Gerätedaten erforderlich
sind \cite{homematic_xmlrpc}.
In homogenen HomeMatic-Umgebungen ist es üblich, dass Verbindungen zwischen Ihnen besteht.
Zum Beispiel kann ein Taster mit Wippe mit einem Aktor gekoppelt sein, um eine Lampe zu dimmen.
Die Art und Beschreibung der Teilnehmer dieser Verbindung kann ebenfalls über eine
bereitgestellte Methode ermittelt werden.

Der Connector muss in der Lage sein die für die Datenerfassung und Steuerung von Geräten benötigten
XML-RPC abzusetzen.
Dazu muss eine Verbindung zur HomeMatic-CCU aufgebaut werden.
Wie in \ref{gru_hm_ccu} beschrieben, existieren grundsätzlich 2 Verbindungsmöglichkeiten, jedoch
dient nur die Netzwerkverbindung zur betriebssystemunabhängigen Verwendung der Programmierschnittstelle.

Für das Verwenden von XML-RPC existieren für beliebige Programmiersprachen viele Bibliotheken.
Der Verbindungsaufbau und das Verschicken und Empfangen einer HTTP-Nachricht ist jedoch in
den meisten Programmiersprachen leicht implementierbar, so dass auf den Overhead bei der Benutzung
einer Bibliothek besteht evtl. verzichtet werden kann.
Die XML-RPC Nachrichten, die von der Software gesendet werden müssen, sind sehr klein und einfach
aufgebaut, so dass kein komplexes Marshalling benötigt wird.

\section{Abbildung der Gerätedaten}
\label{ana_abb}

Zur Interpretierung und Umwandlung der Gerätemodelle von WieDAS und HomeMatic, muss der Connector
die beiden Modelle zunächst in die Implementierungssprache abbilden.
Dazu liegen Schnittstellenbeschreibungen bzw. Spezifikationen bereit.

Die von HomeMatic-Geräten bereitgestellten Daten sind sehr detailliert und oft aus vielen kleineren
Daten zusammengesetzt.
Gerätedaten von WieDAS sind in der Regel einfacher (Kapitel \ref{ana_wd_modell}), da sie viele
herstellerspezifische Gerätedaten mit einem geringen Modell abdecken müssen, da die Speicherkapazität
auf den Systemen, auf denen die Plattform läuft, stark begrenzt ist.
Dies hat zur Folge, dass HomeMatic-Gerätedaten auf ein wesentliches Minimum gekürzt werden müssen und
eventuell komplexere Funktionalitäten auf mehrere kleinere umgesetzt werden müssen.

Dadurch, dass die Spezifikation für das WieDAS-Gerätemodell und für HomeMatic-Geräte nur momentan sind,
ist es erstrebenswert einen Generator zu entwickeln, welcher zumindest einfache Funktionalitäten
automatisch abbilden kann.
Mit dieser Art der Generierung ist es einfacher neue HomeMatic-Geräte einzubinden oder bereits vorhandene
Geräte, welche nicht von WieDAS unterstützt werden zu unterstützen.
Ebenfalls sollte angestrebt werden, dass neue Geräte erst zur Laufzeit dem Connector mitgeteilt werden.
Deshalb ist es notwendig, dass Geräte Informationen bereitstellen, um sie dynamisch anbinden zu können.

Die HomeMatic-Geräte werden häufig durch Strings beschrieben, welche im WieDAS-Datenraum nicht
verwendet werden.
Sollte für die Kommunikation mit der HomeMatic-CCU zusätzliche Informationen zu den Geräten benötigt
werden, die mit den Strings beschrieben werden, so muss der Connector ein eigenes Mapping vorbehalten,
und mit den weiteren Informationen erweitern.

Da die Geräte in WieDAS mit spezifizierten Strukturen beschrieben werden und die HomeMatic-Geräte
als Struktur vorliegen, mit den Namen als Schlüssel und einem variablen Wert, so können einfache
Messdaten identisch übernommen werden.
Sollten Messwerte den Wertebereich eines WieDAS-IDL-Datentypen überschreiten, so sollte der Maximalwert
gewählt werden.
Falls ein HomeMatic-Gerät Funktionen besitzt oder ein Mehrfachgerät ist, müssen für Spezialwerte und
Funktionalitäten separate Strukturen aus der WieDAS-IDL gewählt werden.

Da der AAL-Cache keine Information über die anwendungsspezifische Daten besitzt (hier WieDAS-Gerätemodell)
müssen die Metadaten dazu verwendet werden um das Gerät zu ermitteln.
Denkbar wäre auch eine Struktur zu definieren, die das Gerät kapselt und die Geräteinformation
selbst bereitstellt.

\section{Arbeitsweise des Connectors}
\label{ana_connector}

In der Implementierung müssen verschiedene Funktionen bereitgestellt werden, die der AAL-Cache
aufruft, um den Connector zu registrieren und ihm mitzuteilen, dass ein Tag nicht länger
benutzt werden kann.
Diese Eintrittsfunktionen werden dem AAL-Cache zusammen mit dem Namen der erstellten dynamischen
Bibliothek über eine Konfigurationsdatei mitgeteilt.

Der Connector muss beim Initialisieren oder zeitnah die gewünschten Geräte in den Cache
eintragen.
Dazu ist es notwendig eine Verbindung über das Netzwerk aufzubauen.
Daher sollten nötige Steckverbindungen und Netzwerkeinstellungen bereits vor dem Start
des Connectors vorhanden sein.

Dann werden die Gerätebeschreibungen aus der HomeMatic-CCU gelesen, für WieDAS umgewandelt
und schließlich in den Cache eingetragen.
Um auf neue Geräte und Ereignisse der Geräte zu reagieren, muss sich der Logikprozess (Connector)
bei dem Schnittstellenprozess (CCU-XML-RPC) registrieren.
Das bedeutet auch, dass der Connector einen eigenen XML-RPC Server bereitstellen muss, welcher
die Methoden bereitstellt, die von der HomeMatic Zentrale aufgerufen werden.
Die registrierte Logik-Schnittstelle wird mittels dem Aufruf einer solchen Methode darüber
informiert, dass neue Geräte an der HomeMatic-CCU angemeldet wurden bzw. Aktionen ausgeführt
wurden.
Dieser Aufruf übermittelt die gleichen Informationen, wie das explizite Anfordern
der Gerätedaten bei der ersten Verwendung \cite{homematic_xmlrpc}.

Aus Richtung des AAL-Cache werden ebenfalls Aktionen bzw. Veränderungen mitgeteilt.
Dem Connector (Subscriber) wird vom AAL-Cache mitgeteilt, welche Tags sich geändert haben.
Ihm wird in der registrierten Funktion zwei opake Daten übermittelt.
Ein Datum ist für die Zugehörigkeit im Set zuständig und nicht für eine weitere Verwendung
vorgesehen.
Das andere Datum ist vom Programmierer beim Registrieren der Benachrichtigungsfunktion übergeben
worden.
Der Connector muss dann mit der AAL-Cache Schnittstelle eine Kopie auf das Datum im Cache
anfordern, welches er bereits vorher hinzugefügt hat.

\section{Laufzeitumgebung der Software}
\label{ana_laufzeitumgebung}

Da die Connectoren über dynamische Bibliotheken eingebunden werden, muss die Software
auf einem UNIX-artigen System lauffähig sein (Mac, Linux, BSD, usw.).
Da sich bereits vorhandene Connectoren und somit auch der Cache auf eingebetteten Geräten
etabliert haben und die HomeMatic-CCU ein solches ist, muss der Connector die Anforderungen
aus so einer Umgebung berücksichtigen.
Darunter fällt:
\begin{enumerate}
\item geringe Speichernutzung
\item kleinere, leicht optimierbare Subroutinen
\item Verzicht auf komplexe Datenstrukturen, die den Connector unnötig vergrößern
\item Energieeffizienz (in Bezug auf Threading, Polling, Sleep-State, usw.)
\end{enumerate}

Um die Speichernutzung gering zu halten, sollten redundante Informationen so weit es geht
verworfen oder reduziert werden, ohne dass die Reaktionszeit des Systems stark beeinträchtigt wird.
Es ist zum Beispiel nicht gewollt auf einen Schaltbefehl 2 Sekunden lang warten zu müssen
bis die Lampe erleuchtet, weil das System viele Daten ermitteln muss, um den Schaltbefehl zu realisieren.

Kleinere Subroutinen sind für Compiler aus der Embedded-Umgebung, welche evtl. nicht die
Optimierungsfähigkeiten größerer Compiler besitzen, zu optimieren.
Optimierter Code lässt das System schneller, effizienter und speicherschonender werden.

Da der AAL-Cache dauerhaft auf dem System ausgeführt wird, ist es wichtig, wenig Gebrauch
von globalen Informationen zu machen und stattdessen so viel wie möglich Datenstrukturen
lokal zu halten.
Sollte eine längere Pause oder Sleep-State eingerichtet werden, so dass der Prozess oder
Thread erst bei einer Aktion wieder tätig werden muss und in der Zeit andere Prozesse
arbeiten können.

Falls eine Netzwerkverbindung getrennt wird oder die CCU nicht zu erreichen ist, muss dies
in WieDAS erkennbar sein.
Eine Möglichkeit wäre die Beendigung des Connectors oder das Nutzen der Management-API,
um Fehler bekannt zu geben.

\section{Konfiguration}
\label{ana_konfiguration}

Da der Connector verschiedene Informationen für die Verbindung zur HomeMatic-CCU benötigt,
muss dem Connector die Netzwerkkonfiguration des Systems mitgeteilt werden.
Als Alternative kann der Connector beim Initialisieren selbst Netzwerkeinstellungen,
blockierte Ports, usw. mittels der Betriebssystem-API ermitteln.

Um die Logik-Schnittstelle aufzubauen, benötigt die Software einen freien Port und
eine zugewiesene IP-Adresse, an dem der Server bereitgestellt wird.
Aus Sicherheitsgründen können hier Einschränkungen vorgenommen werden bezüglich
der Reichweite der IP-Adresse und der Portnummer.

Weiterhin ist der Port und die IP-Adresse der XML-RPC Schnittstellen der HomeMatic-CCU
dem Connector mitzuteilen.
Hierbei ist es auch entscheiden, ob auch drahtgebundene Geräte und interne Geräte
beobachtet werden sollen (siehe Kapitel \ref{gru_hm_ha}).

Die HomeMatic-CCU selbst muss ebenfalls zugänglich gemacht werden, da Geräte an ihr
angemeldet werden müssen.
Falls eine enge Kapselung der CCU an das System besteht (z.B. durch ein privates
kleines Netzwerk) muss eine Weiterleitung eingerichtet werden.

HomeMatic bietet mit \emph{NetFinder} eine eigene Software an, um eine oder mehrere
HomeMatic-CCU im Netzwerk zu erkennen.
Falls die Software oder die Vorgehensweise der Software eingesehen werden kann,
könnte dieser Teil der Connector-Konfiguration übersprungen werden und dynamisch
vorgenommen werden.

Weiterhin müssen die Konfigurationsanforderungen des AAL-Cache
(siehe Kapitel \ref{gru_aalcache}) erfüllt werden.
Da die Standardkonfiguration von AAL-Cache für genau die Systeme
gedacht ist, auf denen der Connector lauffähig sein soll.

\section{Designentscheidung}
\label{ana_entscheidungen}

Aufgrund der Einsatzumgebung und der Schnittstellen wäre die Verwendung einer höheren
Programmiersprache unnötig.
Aufgrund der bisherigen Programmiererfahrungen fällt die Wahl zwischen C und C++, wobei
C++ keine entscheidenen Nachteile für die Implementierung mit sich zieht.
Daher wird als Implementierungssprache C++ verwendet.
%TODO Standard

Der Connector selbst wird mit Makefiles übersetzt und gelinkt.
Sollten weitere Prozesse nötig sein, so werden diese auch mittels Makefile-Targets
ausgeführt.
Wie in Kapitel \ref{ana_sw} bereits ermittelt, werden manche Connectoren mit Makefiles
übersetzt.
Daher können diese als Vorlage verwendet werden.
Das Programm \emph{make} mit verschiedenem Funktionsumfang auf fast jedem Betriebssystem
zur Verfügung steht und dem Programmierer geläufig ist.

Für das Absetzen und Bereitstellen von XML-RPC verwendet der Connector die Bibliothek
\emph{xmlrpc-c} \cite{xmlrpc-c}.
Diese Bibliothek ist leichtgewichtig (auf einem derzeitigen Ubuntusystem haben alle Teilbibliotheken
eine Gesamtgröße von 840 KB) und ist in der Lage mit wenigen Aufrufen XML-RPC abzusetzen
und einen Server zu Starten, der die gewünschten Methoden anbietet.

Das Generieren der Schnittstellen und Datenstrukturen für HomeMatic-Geräte findet mittels
einem Generator statt.
Dieser Generator betrachtet zunächst nur Geräte, die zum Zeitpunkt der Entwicklung
zugänglich sind.
Dazu werden die notwendigen Informationen aus der HomeMatic-Dokumentation \cite{hmscript2,hmscript4}
in Textdateien verfasst.
Diese wird dann vom Generator gelesen und erstellt für C++ notwendige Funktionen und
Datenstrukturen.
Da C++ Klassen besitzt, werden die Geräte in jeweils eine Klasse abgebildet.

Im Rahmen der Thesis wird die Entwicklung mit einer statisch adressierten HomeMatic-CCU
und an einem festen Arbeitsplatz durchgeführt.
Daher kann die IP-Adresse und der Port in die Konfigurationsdatei des AAL-Cache geschrieben werden,
so dass der Connector diese beim Initialisieren übergeben bekommt.

