\chapter{Analyse}
\label{analyse}

Das Hauptziel der zu entwickelnden Software ist die Bereitstellung und Verwendung von HomeMatic-Hausautomationsgeräten
im WieDAS-Datenraum.
Im vorherigen Kapitel wurden die Eigenschaften der verwendeten Komponenten, die in der Entwicklung berücksichtigt werden müssen,
beschrieben.
Wie im Abschnitt \ref{gru_aalcache} bereits aufgeführt, wird der WieDAS-Datenraum (Abschnitt \ref{gru_wiedas_daten})
über eine Instanz einer Cache-Software bereitgestellt.
Die HomeMatic-Geräte (Abschnitt \ref{gru_hm_obj}) sind über eine Zentralkomponente (Abschnitt \ref{gru_hm_ccu}) mit dem Verarbeitungssystem verbunden.
Es wird ein Connector entwickelt, der die HomeMatic-Geräte an den Cache koppelt und bestimmten Anforderungen
gerecht werden muss.
In der Analyse ergeben sich die Problemfelder aus den Anforderungen.
Für diese Problemfelder werden Lösungsansätze und Alternativen vorgeschlagen, sowie ihre Vor- und
Nachteile aufgezeigt.
Danach wird der Entwicklungsprozess des Connectors beschrieben und welche Maßnahmen notwendig sind, um den Connector
im Zielsystem zur Ausführung zu bringen und ihn entsprechend zu konfigurieren.

\section{Anforderungen}
\label{ana_anforderungen}

Für die Analyse wurden die folgenden Kern-Anforderungen für den Connector spezifiziert:
\begin{enumerate}
\item Der Connector muss in der Lage sein, das Anmelden und Abmelden von Geräten zu erkennen, um diese
nicht länger im Verarbeitungssystem zu betrachten
\item Gerätefunktionen müssen wahrgenommen werden, so dass sie in den WieDAS-Datenraum eingepflegt
werden können
\item Bei Änderungen im WieDAS-Datenraum muss der Connector dafür sorgen, dass die entsprechende
Gerätefunktion beim HomeMatic-Gerät durchgeführt wird
\item Es werden Verknüpfungen zwischen verschiedenen WieDAS-Daten erstellt. Der Connector muss
eine Möglichkeit bieten diese anzulegen.
\item Da es stetig mehr HomeMatic-Geräte mit neuen Kanaltypen gibt, soll der Connector in der Lage
sein, zu einem späteren Zeitpunkt neuartige Geräte dynamisch zur Laufzeit des Connectors anzulernen.
\item Durch die Verknüpfung der Geräte bietet sich ebenfalls die Möglichkeit, den Connector in heterogenen
WieDAS-Umgebungen lauffähig zu machen und dort auf Geräteaktionen zu reagieren, die von anderen
Connectoren in den Cache geschrieben werden.
\end{enumerate}

Aus den Kern-Anforderungen und der Ausarbeitung der Grundlagen in Kapitel \ref{grundlagen}
gehen weitere funktionale Anforderungen hervor.
Für das Anlernen von Geräten muss der Anwender auf eine Prozedur zurückgreifen, die in
der Beschreibung der Geräte vorliegt.
Es muss analysiert werden, ob Alternativen zu dieser Prozedur angeboten werden oder
realisierbar sind.
Der Connector wird nach der Anmeldung mittels der angebotenen XML-RPC-Schnittstelle
\ref{gru_hm_ccu} über neue Geräte informiert und ist somit nur reagierend.
Auch das Reagieren auf und das Auslösen von Ereignissen von HomeMatic-Geräten wird
über die angebotene XML-RPC-Schnittstelle realisiert.
Die korrekte Benutzung der Schnittstelle und das Verhalten in verschiedenen Situationen
muss analysiert werden.
Bei der Betrachtung der Gerätebeschreibungen in Abschnitt \ref{gru_hm_obj} und Abschnitt
\ref{gru_wiedas_daten} stellt sich heraus, dass die Geräte durch die Unterschiedlichkeit
der Datenstrukturen aufeinander abgebildet werden müssen.
Dabei muss der Wertebereich der Datentypen betrachtet werden und analysiert werden, ob
eine reine Datenabbildung ausreicht oder ebenfalls funktionale Elemente erforderlich sind,
um das gewünschte Verhalten zu erzielen.
Gerade in Bezug auf die Möglichkeit, dass ein HomeMatic-Gerät mehrere WieDAS-Funktionalitäten
ausüben kann, muss eine modulare Grundlage für die Abbildung gefunden werden.
Da die abgebildeten HomeMatic-Geräte dann auch in WieDAS übertragen werden sollen,
muss analysiert werden, wie sie im AAL-Cache identifiziert werden können.
Diese Information muss dann auch verwendet werden, um von Ereignissen aus dem AAL-Cache
das korrekte HomeMatic-Gerät anzusprechen.
Um dem System unbekannte Gerätetypen zur Laufzeit bekannt zu geben, benötigt es einer Geräteverwaltung.
Die Geräteverwaltung muss in diesem Fall benachrichtigt werden und entsprechend die Gerätedaten laden.
Um Verknüpfungen zwischen verschiedenen HomeMatic-Geräten herzustellen, kann man die
HomeMatic-Zentrale nutzen oder schon im Anmeldungsprozess der Geräte eine Verknüpfung herstellen.
Der Connector soll aber auch in heterogenen Umgebungen mit verschiedenen Connectoren
agieren.
Damit Verknüpfungen in heterogenen Umgebungen erstellt werden können, müssen diese
auf WieDAS-Ebene durchgeführt werden.
Daher muss analysiert werden, ob alle WieDAS-Daten verarbeitet werden können.
Es muss analysiert werden, wie das System diese Verknüpfungen herstellen kann und wie die
Möglichkeit dem Anwender angeboten wird.
Dadurch, dass der Connector in eingebetteten Systemen verwendet wird, müssen nicht-funktionale Anforderungen erfüllt werden:
\begin{enumerate}
\item geringe Speichernutzung
\item kleinere, leicht optimierbare Subroutinen
\item Verzicht auf komplexe Datenstrukturen, die den Connector unnötig vergrößern
\item Energieeffizienz (in Bezug auf Threading, Polling, Sleep-State, usw.)
\end{enumerate}

\section{Anwendungsfälle}
\label{ana_uc}

Für die Anwendungsfälle wird davon ausgegangen, dass der Connector im Zielsystem ausgeführt wird
und das System zur Verarbeitung bereit ist.
Der Anwender interagiert also mit dem voll funktionsfähigen System.
Das bedeutet unter anderem, dass
\begin{itemize}
\item Die Netzwerkeinstellungen korrekt sind
\item Der Connector in der Konfigurationsdatei korrekt eingetragen wurde (Abschnitt \ref{gru_aalcache})
\item Der Connector die in der Umgebung vorhandenen Geräte verarbeiten kann
\item Die Verknüpfungen mit der bereitgestellten Möglichkeit hergestellt hat
\item Die notwendigen Informationen zur An- und Abmeldung der Geräte im Besitz des Anwenders sind
\end{itemize}

\begin{figure}[h]
\includegraphics[scale=0.5]{images/uc}
\caption{Anwendungsfälle}
\label{abb_uc}
\end{figure}

%TODO KROEGER (anwendungsfälle nicht besonders aussagekräftig, interoperabilität konkreter als beispiel [eins sensor, eins aktor], möglichst dicht an testfällen)
\subsection{Geräte an- und abmelden}
Der Anwender muss in der Lage sein die Geräte an- und abzumelden.
Dazu wird die Anmeldungsprozedur der Gerätebeschreibung entnommen und an der HomeMatic-Zentrale
durchgeführt.
Der Connector muss diese An- oder Abmeldung entgegennehmen und die Verarbeitung für diese
Geräte einstellen.
Eine Statusmeldung kann dem Anwender darüber informieren, ob die An- bzw. Abmeldung erfolgreich
war.

\subsection{Gerätefunktion auslösen}
Der Anwender kann Funktionen auf dem HomeMatic-Gerät durchführen und diese dann
im AAL-Cache als WieDAS-Datenobjekt wiederfinden, falls das Gerät vom Connector
unterstützt wird.
Dazu kann ein Connector verwendet werden, der alle gewollten URIs beobachtet und
entsprechend ihrer Funktionalität eine einfache Ausgabe tätigt.

\subsection{Geräte verknüpfen}
Weiterhin will der Anwender Verknüpfungen zwischen verschiedenen HomeMatic-Geräten im
WieDAS-Datenraum herstellen.
Die Art und Weise, wie diese Verknüpfung erzeugt wird, ist offen.
Der Aufwand für die Erstellung der Verknüpfung soll im überschaubaren Rahmen liegen und
dokumentiert werden.
Der Benutzer eines anderen Connectors ist in der Lage, mittels der hier entwickelten Geräteabbildung
Verknüpfungen mit den HomeMatic-Geräten herzustellen.

\section{Verwaltung der Gerätemengen}
\label{ana_gemenge}
WieDAS-Datenobjekte sind im Grunde nur Datenvariablen mit Schreib- und Lesefunktionen (Abschnitt \ref{gru_wiedas_daten}).
Sie besitzen wenig Attribute und lassen wenig Spielraum bei der Interpretation der Daten.
Die Recherche im Dokument für Datenpunkte der HomeMatic-Geräte \cite{hmscript4} ergab keinen
Bedarf für eine Abbildung von einer einzelnen Funktion zu mehreren HomeMatic-Geräten.
Dennoch ist es möglich, dass z.B. eine Verknüpfung erstellt wird, wo 2 Taster gleichzeitig
gedrückt werden müssen, um eine Steckdose zu schalten (z.B. aus Sicherheitsgründen).
Sollte diese Art der Verknüpfung erwünscht sein, muss beachtet werden, dass die Ereignisse
dem Connector zeitlich versetzt mitgeteilt werden können.
Dadurch muss der Connector, falls ihm durch das Callback nur das Drücken eines Tasters
mitgeteilt wird, prüfen, ob der andere ebenfalls gedrückt wurde.

In Abschnitt \ref{gru_hm_obj} wird erwähnt, dass HomeMatic-Geräte die Möglichkeit besitzen,
mehrere Funktionen mit einem einzigen Gerät auszuführen (Beispiel Thermostat).
Deshalb muss untersucht werden, welche HomeMatic-Geräte in der Lage sind, welche Funktionen
auszuüben.
Für den Entwurf bzw. die Implementierung sollten alle zur Verfügung stehenden Geräte
berücksichtigt werden, um auch evtl. mehrere Gemeinsamkeiten zu finden und die Geräte
besser modellieren zu können.
Die Unterstützung von mehreren WieDAS-Funktionalitäten in einem einzigen Gerät hat zudem
zur Folge, dass beim Entwickeln der Geräte-Abbildung berücksichtigt werden muss, dass mehrere
URIs auf ein HomeMatic-Gerät zugreifen werden.
Die genaue Abbildung der URIs wird in den Geräteabbildungen näher analysiert.

Um dem System mitzuteilen, welche Geräte vorhanden sind, müssen diese am System angemeldet werden.
Bei Funktionen aus dem WieDAS-Datenraum besteht keine Notwendigkeit, die Geräte anzumelden.
Sie werden durch den Connector, der die Daten bereitstellt angemeldet.
Durch die Verwendung von AAL-Cache (Abschnitt \ref{gru_aalcache}) kann das Datenobjekt
direkt vor dem ersten Einfügen in den Cache \emph{registriert} werden.
Dazu muss nur ein Tag vom Cache angefordert werden, was immer durchgeführt wird, sobald
ein Datum in den Cache hinzugefügt wird.
Der Tag dient zur resourcenschonenden Identifizierung des URI (Abschnitt \ref{gru_aalcache})
und kann jederzeit wieder durch den AAL-Cache zurück in den URI umgewandelt werden.
Eine Abmeldung bzw. Löschung eines Tags aus AAL-Cache muss nicht durchgeführt werden.
AAL-Cache verdrängt die Tags nach dem \emph{Least-Recently-Used}-Verfahren \ref{gru_aalcache}.
Um auf Änderungen im WieDAS-Datenraum zu reagieren, müssen die abgebildeten HomeMatic-Geräte
mitteilen, für welche URIs sie Benachrichtigungen erhalten wollen.
Mit den bekannten URIs kann die Registrierung eines Callbacks durchgeführt werden.
Dieses Callback liefert dann bei Aufruf die veränderten Tags (und dadurch auch die URIs),
und der Connector kann die entsprechenden Geräte benachrichtigen.
Eine Alternative, um die Modellierung zu entschlacken, wäre das Ablegen aller URIs, die
für irgendein HomeMatic-Gerät interessant sein könnte, in einer Datei.
Diese könnte dann vom Connector gelesen werden und jegliches Ereignis zu einem dieser
URIs würde dazu führen, dass jedes Gerät selbst entscheidet, ob es diesen URI
verarbeiten kann.

Bei HomeMatic werden Geräte zunächst der Zentrale bekannt gegeben (Abschnitt \ref{gru_hm_ccu}).
Die Art- und Weise, wie der Anmeldungsprozess vollzogen wird, hängt dabei stark vom Gerät ab.
Beim Durchlesen der Beschreibungen für verschiedene Geräte (insb. Dimmer und Wandtaster)
konnte man feststellen, dass es verschiedene Anmeldungsprozesse gibt.
Zum Beispiel können im Anlernmodus Geräte direkt miteinander verknüpft werden \cite[Seite 6]{hmdimmer}.
Bei ersten Versuchen hat sich herausgestellt, dass die direkte Verbindung dazu führt, dass
die Ereignisse nicht weiter über die externe Schnittstelle weitergeleitet werden.
Geräte können auch über den Schnittstellenprozess der HomeMatic-CCU mit dem Aufruf von \emph{addDevice}
\cite[Seite 16]{homematic_webui_manual} angemeldet werden.
Dazu wird eine Seriennummer benötigt.
Dies wird aber nur für einige Geräte unterstützt.
Bei erfolgreicher An- und Abmeldung wird dem Logikprozess (Abschnitt \ref{gru_hm_ccu}) die Liste
der involvierten Gerätebeschreibungen mitgeteilt.
Um einen Abgleich der Gerätebeschreibungen zuzulassen (Abschnitt \ref{gru_hm_ccu}),
kann ein kleiner Teil der Beschreibungen für die bekannten Geräte gespeichert werden.
Bei einem Ereignis wird die Adresse des logischen Geräts mitgeteilt.
Wird bei einem Zugriff von der Zentrale auf ein HomeMatic-Gerät zugegriffen, so muss eine interne
Assoziation von Adresse zur richtigen Abbildung gespeichert werden.
Wird durch ein Ereignis aus dem WieDAS-Datenraum auf ein HomeMatic-Gerät
zugegriffen, so muss entweder die Zentrale nach der Gerätebeschreibung für die Adresse abgefragt werden
oder durch den Erhalt der Adresse aus dem URI auf die interne Assoziation zugriffen werden.

%Die Abbildung mehrerer WieDAS-Funktionen auf ein Gerät ist Bestandteil der folgenden Analyse.
Bei der Entwicklung des Connectors müsste eine Proxy-Anwendung entwickelt werden, die die
Befehle für eine Anmeldung ausführt und auf dem System läuft, auf dem sich der Connector befindet.
Für andere Geräte müsste auf die Anmeldung an der Zentrale zurückgegriffen werden.
Aufgrund des hohen Mehraufwands wird bei der Entwicklung die Anmeldung der Geräte durch einen
Benutzer nicht einbezogen und ist alleinige Aufgabe des Anwenders.
Um Funktionen im WieDAS-Datenraum entsprechend der Geräte anzubieten und umgekehrt wird ein URI benutzt.
Damit der Connector die richtige Geräteabbildung findet, wird beim Entwurf auf ein Mapping
von Geräte- und Kanaladresse zur Abbildung gespeichert.
Eine Abbildung von einer WieDAS-Funktionalitäten auf mehrere HomeMatic-Geräte wird durch die
Designentscheidungen möglich gemacht, jedoch nicht exemplarisch entworfen, da dafür keine
Geräte im Rahmen der Entwicklungsphase zur Verfügung stehen.
Der Logikprozess speichert die notwendigen Gerätebeschreibungen für die HomeMatic-Zentrale, um
ihr einen Abgleich zu erlauben, da die Liste nur wenig Speicher verbrauchen wird.
Die Verknüpfungen zwischen verschiedenen Geräten im WieDAS-Datenraum wird über das Erstellen
dynamischer Bibliotheken realisiert.
Das gleiche Prinzip wird angewandt, um Geräte zu laden.
Die Geräte werden in dynamischen Bibliotheken abgebildet und decken jeweils das Laden der Geräte
einer Typenreihe ab.
Diese werden bei Programmstart geladen und können bei Bedarf aus einem Verzeichnis nachgeladen werden.
Um die Bibliotheken nachzuladen, muss ein Signal an den Connector geschickt werden.

\section{Abbildung der Geräte}
\label{ana_abbildung}
Die Kernanalyse beschäftigt sich mit der Abbildung von HomeMatic-Geräten und WieDAS-Funktionen.
HomeMatic-Geräte sind im Grunde ein Container für mehrere Kanäle (Abschnitt \ref{gru_hm_obj}).
Die Datenpunkte und Kanäle eines einzelnen Geräts sind leider nicht in den Beschreibungen
der Geräte spezifiziert, sondern müssen dem Dokument für Datenpunkte \cite{hmscript4} entnommen werden.
Im Grunde sind die meisten Datenpunkte selbsterklärend (\emph{PRESS\_SHORT}, \emph{PRESS\_LONG}, usw.)
jedoch gibt es einige Datenpunkte, wo eine Beschreibung sinnvoll wäre (\emph{INSTALL\_TEST}, \emph{INHIBIT}, usw.).
Man könnte sich vorstellen, dass ein Kanaltyp (z.B. \emph{Key}) oder ein Datenpunkt immer genau für eine
WieDAS-Funktion verwendet werden kann.
Jedoch würde diese 1:1 Verknüpfung keinen Sinn ergeben.

\begin{figure}[h]
\includegraphics[scale=0.45]{images/abb_taster}
\caption{Mögliche Abbildungen eines HomeMatic-Tasters in WieDAS}
\label{abb_taster}
\end{figure}

Abbildung \ref{abb_taster} zeigt 2 verschiedene Möglichkeiten einen Taster in WieDAS abzubilden.
Ein Taster besitzt 2 Kanäle vom Kanaltyp \emph{Key} (hier: \emph{K1} und \emph{K2}).
Der Kanaltyp besitzt jeweils die Datenpunkte \emph{PRESS\_SHORT}, \emph{PRESS\_LONG} und
\emph{PRESS\_CONT} \cite{hmscript4} (jeweils vom Datentyp: \emph{BOOL}).
Nun gibt es die Möglichkeit, den Taster entweder so abzubilden, dass ein Kanal der WieDAS-Funktionalität
zugeordnet wird.
Dazu müsste ein kurzer Tastendruck auf einer Seite des Tasters dafür sorgen, dass in der WieDAS-Funktionalität
die Werte \emph{on} oder \emph{off} gesetzt werden und bei einem langen Tastendruck auf der gleichen Seite
des Tasters der entsprechend andere Wert.
Eine andere Möglichkeit ist es, den Taster so abzubilden, dass ein kurzer Tastendruck auf einer Seite
die Funktionalität \emph{on} und ein kurzer Tastendruck auf der anderen Seite das Attribut in der
WieDAS-Funktionalität auf \emph{off} setzt.
Beim Entwurf der Geräte können beide Möglichkeiten berücksichtigt werden.
Dadurch ist es möglich, dass ein Taster (mit 2 Druckzuständen pro Seite) 2 WieDAS-Funktionen anbieten kann.
Das Beispiel verdeutlicht auch, dass es keine allgemeine Abbildungsvorschrift gibt.
Die Geräte müssen einzeln entworfen und implementiert werden.
Es kann jedoch beim Entwurf schon für beide Szenarien eine entsprechende Abbildung unter einem anderen
Namen erstellt werden (z.B. Taster und Taster-2).

%TODO KROEGER (allgemeiner, wenn vorher cache-einträge als structs formal dann hier z.b. id, typ, etc.)
Geräte müssen in der Lage sein die angebotenen Funktionalitäten in URIs zu beschreiben.
Eine Möglichkeit, die URIs zu formen, ergibt sich aus der WieDAS-IDL \cite{wiedas_idl}
und den verschiedenen Kanälen und deren Typen.
Die Identifikation von HomeMatic-Geräten geschieht über Adressen (Abschnitt \ref{gru_hm_obj}).
Um der URI eine Information zukommen zu lassen, dass es sich um ein Gerät der HomeMatic-Produktserie,
kann dem URI eine Zeichenkette hinzugefügt werden.
Dies ist nicht zwingend erforderlich, hilft aber dabei, die einzelnen Geräte im WieDAS-Datenraum
zu unterscheiden.
Eine weitere Möglichkeit wäre es, dem URI den Aufenthaltsort des Geräts mitzuteilen, jedoch kann
dadurch bei Auftreten eines Ereignisses nicht direkt herausgefunden werden, welches Gerät
an einem Ort gemeint ist (Beispiel: 2 Taster in der nähe eines Türrahmen), sie müssten zusätzlich
mit einem Index versehen werden.
Weiterhin könnten Aufgaben als Identifikator benutzt werden.
Für die beiden letzten Abbildungsmöglichkeiten bedarf es der Speicherung einer Abbildung
von Aufenthaltsort/Funktion zur eigentlichen Adresse.

\begin{table}[h]
\begin{tabular}{|l|l|l|l|l|}
\hline
Gerät & Kanal\# & Aufgabe & Ort & URI \\
\hline
ABC123 & 1 & Licht & Schlafzimmer & /OnOffFunctionality/HMABC123-1 \\
\hline
ABC123 & 1 & Licht & Schlafzimmer & /OnOffFunctionality/Schlafzimmer1 \\
\hline
ABC123 & 1 & Jalousie & Schlafzimmer & /OnOffFunctionality/Jalousie1 \\
\hline
\end{tabular}
\caption{Möglichkeiten der URI-Abbildung}
\label{tab_abb_uri}
\end{table}

Tabelle \ref{tab_abb_uri} zeigt Möglichkeiten für die Abbildung von URIs.

Da die Hashing-Funktion für die Erzeugung von Tags aus URIs in AAL-Cache eine breite
Verteilungsfunktion besitzt (Abschnitt \ref{gru_aalcache}), hat die Wahl des URI
keine Konsequenzen in Bezug auf Speichernutzung, weil z.B. zu viele Tags entstehen könnten.

Es gibt 2 verschiedene Verwendungen der Abbildungen.
Für die Identifikation der 2 benötigten Verwendungen dient der URI.
Eine Abbildung wird benötigt, sobald ein HomeMatic-Ereignis, durch Aufruf der
\emph{event}-Funktion (Abschnitt \ref{gru_hm_ccu}) dem Connector unter Verwendung
der Logikschnittstelle mitgeteilt wird.
Wie bereits im Abschnitt \ref{ana_gemenge} ermittelt, wird ein Mapping der Geräte gespeichert.
Daher ist es wichtig zu wissen, welche Informationen für das korrekte Abbilden für das
Gerät benötigt werden.
Eine unnötige Ansammlung von zu vielen Daten für ein Gerät hat einen großen Anstieg
des benötigten Speicherbedarfs zur Folge.
Eine Alternative wäre, die nötigen Informationen von der Zentrale abzurufen, sobald
sie für das Ausführen von verschiedenen Funktionen benötigt werden.
Wird ein Ereignis ausgeführt, gibt es die Möglichkeit, die Daten für das Gerät zusammen
mit einer ``Verhaltensregel'' für entsprechende Daten aus dem Mapping zu erhalten oder
man bestückt den Geräteentwurf mit Funktionen, welche dann auf den eigenen
Daten agiert.

Eine 2. Verwendung für die Abbildung erfolgt, wenn der AAL-Cache das Datum der Funktionalität
erneuert und dem Connector über das registrierte Callback dies mitteilt (Abschnitt \ref{gru_aalcache}).
Die WieDAS-Funktionen besitzen zusätzlich noch eine ID für die eindeutige Identifizierung
im WieDAS-Datenraum (Abschnitt \ref{gru_wiedas_daten}), welche aber nicht benötigt wird.
Um trotzdem eine gewisse Eindeutigkeit zu erreichen, wie es die WieDAS-IDL
für dieses Attribut vorsieht, können Teile der Adresse, z.B. Hash oder
Ausschnitte davon benutzt werden.
Um die IDL bzw. die spezifizierten Datentypen für den Connector zugänglich zu machen, muss die
WieDAS-IDL-Datei für die verwendete Programmiersprache angepasst werden.
Dafür existieren eine Reihe von IDL Präprozessoren.
Es kann jedoch auch ein eigener Präprozessor geschrieben werden oder die Umwandlung manuell
durchgeführt werden, um Abhängigkeiten zu anderen Projekten zu vermeiden.
Durch Verwendung der IDL in anderen Connectoren ist der Connector in der Lage, bei ihnen
Gerätefunktionen auszulösen oder von ihnen zu empfangen.
Damit das Gerät auf eine Änderung im WieDAS-Datenraum reagieren kann, müssen die URIs,
die das Gerät identifizieren, registriert werden.
Daher muss für die Abbildung der URI oder die URIs gespeichert werden und dem Connector
bei der Registrierung der Callback-Funktionen zur Verfügung gestellt werden.
Damit das Gerät dann weiß, um welches WieDAS-Datenobjekt es sich handelt, muss der URI
der Verarbeitung des Ereignisses übergeben werden.

Für das Design habe ich mich entschieden, dass die Geräte durch ihre starke Abweichung von den
Funktionen der WieDAS-Geräte funktional auf die eintretenden Ereignisse reagieren und die
Verarbeitung nicht mittels Daten und Verhaltensregel an eine Verarbeitungskomponente delegiert.
Es wird jedoch ein Generator entwickelt, um die Deklarationen und Verarbeitungsschritte, die immer
benötigt werden, nicht jedes mal neu schreiben zu müssen.
Dieser Generator hilft dann die in Abschnitt \ref{ana_gemenge} angesprochenen dynamischen
Bibliotheken zu erstellen.
Die Geräte selbst speichern die Informationen, die sie zum Verarbeiten der Ereignisse
benötigen, da für das Abrufen der Informationen von der Zentrale zu viel Zeit verloren geht.
Die URIs werden aufgrund der Einfachheit aus der WieDAS-Funktionalität, einer Zeichenkette,
der Adresse und einer ID zusammengesetzt.
Die Zeichenkette dient als Indikator, dass es sich um HomeMatic-Geräte handelt.
Die ID ist für die interne Verwendung der Geräte (z.B. Bündelung von Kanälen).

\section{Verwendung der Schnittstellen}
\label{ana_schnitt}
Der Connector verwendet im wesentlichen 2 Schnittstellen.
Die Schnittstelle zur Anbindung an die HomeMatic-Zentrale (Abschnitt \ref{gru_hm_ccu} und
für die Interaktion mit AAL-Cache (Abschnitt \ref{gru_aalcache}).

Die HomeMatic-Zentrale kommuniziert mit dem Connector über das XML-RPC-Protokoll (Abschnitt \ref{gru_xmlrpc}).
Um das XML-RPC-Protokoll zu benutzen, muss ein XML-RPC-Client eingesetzt werden, der die
Nachrichten erstellt und abschickt.
Der XML-RPC-Client kann dabei synchron oder asynchron arbeiten.
Der Client muss, durch die Forderung des Protokolls, als Transportprotokoll HTTP nutzen.
Eine Alternative, um dies zu unterbinden, wäre die Kommunikation über HTTP über SSL.
Wie in Abschnitt \ref{gru_hm_ccu} bereits erklärt wurde, steht als einzige Sicherheitsmaßnahme für die
Zugriffsrestriktion auf die Schnittstelle die Möglichkeit die IP-Adressen anzugeben, welche Zugriff bekommen.
Dadurch ist die Kommunikation zwischen der HomeMatic-CCU und dem Connector vollständig einzusehen
und manipulierbar.
%TODO DANIEL (strich zwischen Der und Client)
Der Client ist dafür zuständig, die Funktionen, die im XML-RPC-Dokument von HomeMatic \cite{homematic_xmlrpc}
hinterlegt sind, für den Connector zugänglich zu machen.
Die verschickten Nachrichten bestehen vollständig aus Text.
Da die Nachrichten einem bestimmtem Format entsprechen (Abschnitt \ref{aufbau_xmlrpc}) kann
ein Template benutzt werden, um nur die entsprechenden Werte für den Aufruf einzufügen.
Dabei müssen Konvertierungen von Datentypen und Datenstrukturen vorgenommen werden.
Das bedeutet, dass ein Modul entworfen werden muss, welches die Datentypen aus der XML-RPC-Schnittstellenbeschreibung
von HomeMatic in verwertbare Datentypen für den Connector umwandelt.
Gleichzeitig muss das Modul die verwendeten Typen in Text umwandelt, um sie mit XML-RPC über das Netzwerk zu schicken.

Ein wichtiger Bestandteil beim Arbeiten mit dem Client ist das Behandeln von Fehlern beim Empfangen oder
Senden von Nachrichten.
Tritt ein Fehler beim Senden von Nachrichten auf, so sollte der Connector in der Lage sein dies abzufangen
und zu ignorieren.
Ein Abschalten des Connectors bei einer geringen Netzwerkstörung oder bei Fehlfunktion der HomeMatic-Zentrale
wäre nicht angemessen.

Die HomeMatic-CCU stellt 3 XML-RPC-Schnittstellenprozesse zur Verfügung (Abschnitt \ref{gru_hm_ccu}), jedoch werden
bei der Entwicklung des Connectors nur drahtlos angebundene Geräte berücksichtigt, wodurch der Connector
nur mit einem Schnittstellenprozess kommunizieren muss.
Ein Client alleine reicht jedoch nicht, um den Anforderungen gerecht zu werden.
Da auf Ereignisse aus der Zentrale reagiert wird, muss der Connector eine Logikschnittstelle (Abschnitt \ref{gru_hm_ccu})
bereitstellen.
Dazu muss ein XML-RPC-Server im Connector ausgeführt werden.
Der Server sorgt dafür, die nötigen Methoden, die im XML-RPC-Dokument von HomeMatic \cite{homematic_xmlrpc} angegeben sind,
anzubieten.
Die Methoden können dann entweder direkt auf die Ereignisse bzw. Aufrufe reagieren oder aber das Warteschlangenprinzip
nutzen, welches auch vom AAL-Cache benutzt wird.
Dazu könnten einfache Datenstrukturen verwendet werden, die den Namen des Aufrufs und die übergebenen Daten kurzzeitig
speichern.
Der Connector kann dann, sobald er in einen Idle-Zustand verfällt, die relevanten Datenstrukturen betrachten und
die entsprechende Arbeit verrichten.
Falls der Server einen Fehler bei der Verarbeitung des Requests feststellt, so muss er dies dem Aufrufer mitteilen
(Abschnitt \ref{aufbau_xmlrpc}).
Der Schnittstellenprozess hat für Fehler in der XML-RPC-Nachricht entsprechende Fehlercodes definiert (Abschnitt \cite{homematic_xmlrpc}).
Da der Server ständig auf Ereignisse warten muss, ist es notwendig, ein Polling, also eine Abfrage für eingehende
Nachrichten, durchzuführen oder auf eingehende Nachrichten in einem Thread oder Prozess zu warten.

Der AAL-Cache verwendet eine rudimentäre C-Schnittstelle, um auf den Kern zugreifen zu können (Abschnitt \ref{gru_aalcache}).
Beim Eintragen von WieDAS-Daten in den Cache verwendet das Gerätemodell (Abschnitt \ref{ana_abbildung}) die Funktionalität
zum Einfügen von Daten.
Ein größerer Aspekt ist die Verwendung der Callback-Funktionalität von AAL-Cache.
AAL-Cache stellt für die Registrierung eines Callbacks eine Funktion bereit.
Ihr wird mitgeteilt, wie viele Tags und welche Tags beobachtet werden sollen.
Eine Möglichkeit wäre es, den Gerätemodellen die Registrierung ihrer URIs zu überlassen.
Dies würde allerdings zu einem starken Anstieg des Speichverbrauchs führen, da für jede Registrierung eine
Datenstruktur aufgebaut wird (\emph{SubscriptionSet-Handle}) und im AAL-Cache gespeichert wird.
Eine Alternative ist das Registrieren aller gewünschten Tags in einem einzigen \emph{SubscriptionSet}.
Dann müsste der Connector dafür sorgen die URIs, welche im Callback mitgeteilt werden, den Geräten zuzuordnen
und die Geräte darauf reagieren zu lassen (Abschnitt \ref{ana_abbildung}).
Callbacks werden in Threads ausgeführt und man darf keine Annahmen über den Thread treffen, in dem das
Callback aufgerufen wird \cite{aalcache}.
Dadurch das der XML-RPC-Server für das Abarbeiten der Methodenaufrufe (s.o.) ebenfalls Threads oder ähnliches
benutzen muss, müssen Vorkehrungen getroffen werden um geteilte Datenstrukturen zu schützen.
Wie schon bereits im vorherigen Absatz erläutert, benutzt AAL-Cache ein Warteschlangenprinzip zum Einreihen
der Callback-Funktionen.
Um die Abarbeitung im gleichen Prozesskontext bzw. Thread durchzuführen, kann hier wiederum das gleiche
Prinzip angewandt werden.
Das Callback kann die nötigen Abarbeitungen einreihen und im gleichen Kontext ausführen, wie die restlichen
Aufgaben.

Für den Entwurf der Schnittstellenabbildung habe ich entschieden die XML-RPC-Bibliothek \emph{xmlrpc-c} \cite{xmlrpc-c} zu verwenden.
Diese bietet eine Client- und Serverimplementierung, sowie nötige Hilfsmittel zur Generierung von Nachrichten und
 ist durch die Verwendung von C geeignet den Connector ohne Hinzufügen von weiteren Abhängigkeiten zu unterstützen.
Ein striktes Binden an ein Kommunikationsmodell des Clients (synchron oder asynchron) ist anhand der Analyse nicht nötig.
Die Bibliothek bietet sowohl asynchrone als auch synchrone Schnittstellenaspekte an.
Die Serverimplementierung wird allerdings nur synchron angeboten.
Da die Verwendung der asynchronen Client-Schnittstelle jedoch die Komplexität, durch Verwendung von Betriebssystem-abhängigen
Netzwerkschnittstellen, in den Connector einbringen würde, wird die synchrone Schnittstelle verwendet.
Da der Sicherheitsaspekt bei der Kommunikation nicht gefordert wurde, ist es hier nicht nötig, Rücksicht auf die
beschrieben Problematik bezüglich der Sicherheitsprobleme bei ungesicherter Übertragung zwischen den Komponenten zu nehmen.
Diese wären Bestandteil der Analyse einer Weiterentwicklung des Connectors.
Es existieren für die Bibliothek Möglichkeiten, XML-RPC über HTTPs zu übertragen \cite{xmlrpc_ssl}.
Da die Bibliothek sich intern um die Struktur der Nachrichten für Requests und Antworten kümmert, müssen die
analysieren Fehlerbehandlungen auf Protokollebene nicht weiter beachtet werden, da sie abstrahiert werden.
Trotzdem muss darauf geachtet werden, dass Fehler beim Absetzen von Nachrichten zwar erkannt werden, aber nicht
zur Beendigung des Connectors führen, da kurzzeitige Störungen bei der Kommunikation vernachlässigt werden.
Da die HomeMatic-Zentrale keine Aussagen darüber trifft, ob sie eine Fehlerbehandlung durchführt, sobald
der Logikprozess eine Fehlermeldung zurückliefert, gibt es keine Rückmeldung (\emph{faultCode}-Element)
von Fehlern.
Für die Abarbeitung der Callbacks und der Ereignisse der HomeMatic-CCU kann das angesprochene Warteschlangenprinzip
nicht verwendet werden.
Da der XML-RPC-Server der Bibliothek nicht unterbrochen werden kann, müsste die Abarbeitung des AAL-Cache
Callbacks warten, bis der nächste RPC bearbeitet wird.
Deshalb wird Locking verwendet, um die geteilten Resourcen vor gleichzeitigem Zugriff zu schützen.

\section{Entwicklungsprozess}
\label{ana_entwicklung}

%TODO KROEGER (wohl epic fail)
Der Connector ist eng an den AAL-Cache gekoppelt \cite{aalcache} und durch die Verwendung der XML-RPC-Schnittstelle
von den HomeMatic-Geräten entkoppelt.
Für den Entwurf des Connectors wurden mehrere bereits vorhandene Connectoren im AAL-Cache-Code-Repository \cite{aalcache_code}
betrachtet.
Das Repository enthält die Schnittstellen, die Implementierung, eine Dokumentation, sowie verschiedene Connector-Implementierungen
mit Bauanleitungen.
Die Connector-Implementierung wird in der Umgebung des Code-Repository erfolgen, um ein einfaches Hinzufügen der Implementierung
für zukünftige Weiterentwicklung zu ermöglichen.

Die Bauanleitungen benutzen im wesentlichen ANT-Dateien \cite{ant} und Makefiles \cite{make}.
Das Ergebnis der Verarbeitung der Bauanleitung ist mindestens eine dynamisch ladbare Bibliothek.
Durch das Einbinden der dynamischen Bibliothek in den Connector zur Laufzeit ist der Connector eng gekoppelt.
Die dynamische Bibliothek wird dann in einer Textdatei (\emph{connectors.conf}) eingetragen, so dass der Connector diese
beim Starten laden kann.
Für den Entwicklungsprozess stellt der AAL-Cache eine Debugging-Schnittstelle zur Verfügung, welche durch die Verwendung eines
Makros auch wieder deaktiviert werden kann.

Das Verwenden von dynamischen Bibliotheken hat zur Folge, dass ein UNIX-artiges Betriebssystem (Mac, Linux, BSD, usw.) notwendig
ist, um AAL-Cache zu starten und den Connector zu laden.
Da vorhandene Connectoren und der Cache auf eingebetteten Geräten eingesetzt werden, muss der Connector die nicht-funktionalen
Anforderungen, die in Abschnitt \ref{ana_anforderungen} ermittelt wurden, berücksichtigen.

Um die Speichernutzung gering zu halten, sollten redundante Informationen so weit es geht
verworfen oder reduziert werden, ohne dass die Reaktionszeit des Systems stark beeinträchtigt wird.
Es ist zum Beispiel nicht gewollt, auf einen Schaltbefehl 2 Sekunden lang warten zu müssen,
bis die Lampe erleuchtet, weil das System viele Daten ermitteln muss, um den Schaltbefehl zu realisieren.

Kleinere Subroutinen sind für Compiler aus der Embedded-Umgebung, welche evtl. nicht die
Optimierungsfähigkeiten größerer Compiler besitzen, zu optimieren.
Optimierter Code lässt das System schneller, effizienter und speicherschonender werden.

Da der AAL-Cache dauerhaft auf dem System ausgeführt wird, ist es wichtig, wenig Gebrauch
von globalen Informationen zu machen und stattdessen so viel wie möglich Datenstrukturen
lokal zu halten.
Je nach Servermodell müssen evtl. Sleep-States oder Timeouts benutzt werden, um andere
Prozesse bzw. Threads arbeiten zu lassen, während man selbst auf Ereignisse wartet.

Da der Connector verschiedene Informationen für die Verbindung zur HomeMatic-CCU benötigt,
muss dem Connector die Netzwerkkonfiguration des Systems mitgeteilt werden.
Falls eine Netzwerkverbindung getrennt wird oder die CCU nicht zu erreichen ist, muss der Connector
entsprechend darauf reagieren ohne sich abzustellen.
So muss der Anwender bei kleineren Netzwerkstörungen den Connector nicht neu laden.
Als Alternative kann der Connector beim Initialisieren selbst Netzwerkeinstellungen,
blockierte Ports, usw. mittels der Betriebssystem-API ermitteln.
Um die Logik-Schnittstelle aufzubauen, benötigt die Software einen freien Port und
eine zugewiesene IP-Adresse, an dem der Server bereitgestellt wird.
Aus Sicherheitsgründen können hier Einschränkungen vorgenommen werden bezüglich
der Reichweite der IP-Adresse und der Portnummer.
Da die Software und die Vorgehensweise des \emph{NetFinder} Programms (Abschnitt \ref{gru_hm_ccu})
nicht eingesehen oder erahnt werden kann, muss dem Connector die IP-Adresse und
dem Port der XML-RPC-Schnittstelle der HomeMatic-CCU mitgeteilt werden.
Dies kann durch die Konfigurationsdatei \emph{connectors.conf}-Datei geschehen.
Falls eine enge Kapselung der CCU an das System besteht (z.B. durch ein privates
kleines Netzwerk) muss eine Weiterleitung eingerichtet werden.
