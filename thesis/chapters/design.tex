\chapter{Design}
\label{design}

Dieses Kapitel zeigt den Entwurf des HomeMatic-Connectors für den WieDAS-Datenraum.
Der Entwurf erfüllt die grundlegenden Anforderungen, die im letzten Kapitel
herausgearbeitet wurden und wurde in den Entscheidungen begründet.
Als erstes wird der Entstehungsprozess des Connectors erläutert.
Dann folgt eine Beschreibung der Gesamtarchitektur, in welcher sich der Connector befindet
und eine Kurzbeschreibung der Komponenten mit denen er über verschiedene Kommunikationswege
interagiert.
Komplexere HomeMatic-Gerätedaten, welche vom XML-RPC-Protokoll übertragen und
entsprechend verarbeitet werden müssen, besitzen eine Abbildung in Klassen zur einfacheren
Nutzung, welche anhand von Beispielen beschrieben wird.
Danach wird die Geräteabbildung zwischen HomeMatic und WieDAS, die aus (Abschnitt \ref{ana_abbildung})
hervorgeht erläutert und auf die die Modularität der Geräte eingegangen, um zu zeigen,
wie Geräte entworfen werden können.
Beispielhaft wird der Aufbau von 2 abgebildeten Geräten (Steuerung und Aktor) beschrieben.
Anschließend werden die relevanten Schnittstellen im Connector abgebildet (Abschnitt \ref{ana_schnitt})
und gezeigt, wie die Schnittstellen im Connector miteinander interagieren.
Darin sieht man die verwendete XML-RPC Bibliothek und ihre Auswirkungen auf den Entwurf.
Da nicht alle Gerätetypen, die HomeMatic unterstützt, modelliert werden,
müssen unbekannte Gerätetypen zu einem späteren Zeitpunkt angelernt werden.
Dazu wird eine Geräteverwaltung \ref{ana_gemenge} entwickelt, die die Bibliotheken laden kann.
Diese dient auch für das Nachladen von Verknüpfungsmodulen, um Geräte miteinander
interagieren zu lassen.
Der Aufbau solch einer Verknüpfung wird zunächst allgemein und dann anhand der Verknüpfung
zwischen den beiden beispielhaft entworfenen Geräte erläutert.

%\pagebreak

\section{Entstehungsprozess}
\label{des_entstehung}
Als erstes muss die WieDAS-IDL in eine für die Programmiersprache benutzbare Quelle
umgewandelt werden.
Daraufhin können Geräteabbildungen und Geräteverknüpfungen entwickelt werden.
Diese befinden sich dann in Bibliotheken, welche zur Laufzeit geladen werden können.
Um diese zu erstellen wird die Entwicklungsumgebung genutzt, welche schon für das
Erstellen von Bibliotheken vorbereitet wurde (Abschnitt \ref{ana_entwicklung}).
Da die Geräte nicht auf einem grundlegendem Modell basieren, sondern nur bestimmte Attribute
untereinander teilen, müssen diese eigenhändig programmiert werden.
Geräteverknüpfungen werden nur über die WieDAS-Daten vorgenommen.
Dadurch sind Geräteverknüpfungen vergleichsweise einfacher zu erstellen.
Da es aber auch die Möglichkeit geben soll die Verknüpfung zu programmieren,
wird sie ebenfalls als dynamische Bibliothek erzeugt.
Die Verknüpfungen selbst haben selbst dann keine Bindung mehr an die HomeMatic-Geräte.
In einer dynamischen Bibliothek können mehrere Geräteverknüpfungen bzw.
Geräteabbildungen vorhanden sein.
Als Ergebnis des Prozesses liegt am Ende eine dynamische Bibliothek für
den AAL-Cache und beliebig viele Bibliotheken für Verknüpfungen und Geräte vor.

\section{Gesamtarchitektur}
\label{des_gesamt}

\myfigure{h}{design_gesamt}{Gesamtarchitektur mit Kommunikationswegen}

In der Abbildung \ref{abb_design_gesamt} sieht man die Komponenten, die im System
vorkommen und ihre Kommunikationswege.
Der WieDAS-Datenraum befindet sich in den Daten des AAL-Cache Programms.
AAL-Cache ist das Programm, welches den Connector lädt und im eigenen Prozess mit
einem Thread lauffähig macht (Abschnitt \ref{gru_aalcache}).

Zwischen den einzelnen Komponenten befinden sich verschiedene Kommunikationswege.
Aufgrund von geteilten Daten und asynchronem Verhalten, welches durch die
Verwendung dieser entstehen, müssen sie im Entwurf berücksichtigt werden.

Der HomeMatic-Connector benutzt die Schnittstellen des AAL-Cache, um Gerätedaten
von HomeMatic-Geräten in den WieDAS-Datenraum zu übertragen.
Ebenso können andere Connectoren im AAL-Cache-Prozess vorhanden sein, welche
die HomeMatic-Geräte steuern wollen.
Dadurch muss der Connector auf Ereignisse entsprechend reagieren.

Im wesentlichen benutzt der Connector die HomeMatic-Zentrale, welche dafür verantwortlich
ist neue Geräte in das System aufzunehmen, vorhandene zu konfigurieren oder zu
verbinden.
Die HomeMatic-Zentrale ist dabei über das Netzwerk mit dem Rechensystem verbunden und
verwendet für das Abrufen und Mitteilen das XML-RPC-Protokoll (Abschnitt \ref{gru_xmlrpc}).
Dies hat zur Folge, dass ein XML-RPC-Client als auch Server, die durch eine Bibliothek
geliefert werden (Abschnitt \ref{ana_schnitt}), in den Entwurf mit eingeplant werden müssen.

Der Anwender ist in der Lage über das Netzwerk mit einen Web-Browser auf die Zentrale
zuzugreifen, um spezifische Gerätekonfigurationen vorzunehmen.
Die Zentrale selbst kann mit Geräten über Kabel, sowie über Funk kommunizieren
(Abschnitt \ref{gru_hm_ccu}).
In diesem Entwurf wird die Kommunikation über Kabel nicht berücksichtigt (Abschnitt \ref{ana_schnitt}).

Bei der Modellierung der Geräte und dem Entwurf der Geräteabbildungen muss berücksichtigt
werden, dass die Abbildung in der Lage ist, sowohl aus WieDAS-Daten das passende
HomeMatic-Gerät anzusprechen und umgekehrt (Abschnitt \ref{ana_abbildung}).

\section{HomeMatic-Datenabbildung auf XML-RPC}
\label{des_xmlrpc_abbildung}

Für die Kommunikation mit der HomeMatic-Zentrale, also der Übertragung von Parametern
und Daten und dem Erhalt von Geräteinformationen, wird eine XML-RPC-Bibliothek verwendet.
Diese verwendet einen opaken Datentyp für jeden übertragenen Wert.
Dieser speichert zusätzlich eine Information darüber welchen spezifischen Typ er kapselt.
Innerhalb des Connectors wird er in einen Typen umgewandelt, der die Nutzung vereinfacht.
Die Abbildung primitiver Typen wird bei Bedarf in den jeweiligen Aufrufen bzw. Antworten
realisiert, da die Bibliothek dafür einfache Operationen anbietet.

Für übertragene Strukturen (hier: spezifische HomeMatic-Daten) existiert ein Modul,
welches die XML-RPC Daten entsprechend der HomeMatic-Dokumentation interpretiert und
jeweils als Klasse abbildet.
Es wird eine Basisklasse genutzt, welche mit Hilfe der übertragenen Nachricht erstellt
werden kann und die Assoziationen der XML-RPC-Nachricht speichert.
Dabei bilden Zeichenketten, aufgrund der Recherche in der HomeMatic-Dokumentation über Datenpunkte, die Schlüssel
der Assoziation.
Da die Werte der Assoziationen von Datentyp zu Datentyp verschieden sind muss das Datum
einen eigenen \emph{Any}-Typen entwickeln, um die Bibliothek nicht nach außen hin sichtbar
zu machen.
Da dies aber unnötigerweise Abhängigkeiten einbringen würde, wurde darauf verzichtet und
der opake Datentyp der Bibliothek verwendet.
Da der opake Datentyp nach einer ersten Zuweisung nicht mehr zugewiesen werden kann,
bedarf es einer alternativen Speicherung des Wertes oder einer Modifizierunsoperation,
die den kompletten Assoziationseintrag erneuert.

Da bei genauerer Betrachtung der Operationen davon ausgegangen wurde, dass eine Zuweisung,
also Erstellung von Datentypen zur Übertragung an den Schnittstellenprozess, so gut wie
nie vorkommt, muss eine Modifizierungsoperation implementiert und eine einfache Assoziation
von Zeichenkette zum opaken XML-RPC Datentyp gespeichert werden.
Das bedeutet, dass bei einer Modifizierung eines Eintrags der komplette Assoziationseintrag
erneuert werden muss.
Andernfalls hätte man auf eine Speicherung jedes einzelnen typisierten Werts oder
auf die Verwendung einer anderen Bibliothek zurückgreifen müssen, um einen \emph{Any}-Container
zu erhalten.

Die Basisklasse besitzt die Möglichkeit aus dem opaken Wert erstellt zu werden (s.o.), dies
gilt aber nur solange der gekapselte Typ eine Struktur mit Zeichenketten als Schlüssel
besitzt.
Weiterhin hat sie die Möglichkeit sich wieder zurück in einen solchen opaken Wert
zu formen, der dann wieder für die Bibliothek genutzt werden kann.
Für die Umwandlung zwischen primitiven Datentypen und dem opaken Typ gibt es ein
extra Modul mit Umwandlungsfunktionen, der im Grunde genommen nur auf die Umwandlungsoperationen
der Bibliothek zugreift, diese jedoch für einfachere Benutzung abstrahiert.

\vfill
\pagebreak

\myfigure{h}{design_hm_base}{Basisklasse zur Speicherung von XML-RPC-Assoziationen}

Abbildung \ref{abb_design_hm_base} zeigt den Ablauf beim Erstellen von Strukturen für
und das Verarbeiten dieser von XML-RPC Nachrichten.

Die Basisklasse kann dann von den HomeMatic-Daten-Klassen vererbt werden.
Bei der Erstellung von leeren Objekten werden die Assoziationspaare später
einzeln, durch Verwendung der Modifizierungsfunktion, erstellt.
Dies wird dann angewendet, wenn man XML-RPC Nachrichten verschicken möchte.
Bekommt man eine XML-RPC Nachricht, so werden die Objekte vom opaken
Wert der Bibliothek erstellt und werden in der Regel nur noch lesend benutzt.

Vererbende Klassen zusätzliche Operationen oder Attribute anbieten, um die
Nutzung der Wertepaare zu vereinfachen.
Sie können z.B. für schnelleren und einfacheren Zugriff Zeichenketten in Nummern
umwandeln oder die Assoziationspaare interpretieren und als eigenständige Methode
anbieten.
Dadurch dass in der Basisklasse die Assoziation direkt modifiziert werden kann,
müssen vererbende Klassen darauf achten, dass sie keine Zustände der Assoziation
speichern, sondern nur Operationen auf den derzeitigen Stand durchführen.

\vfill
\pagebreak

\myfigure{h}{design_hm_spec}{Vererbung von spezifischen Klassen zur Abstraktion von HomeMatic-Daten}

In Abbildung \ref{abb_design_hm_spec} wird gezeigt, wie verschiedene Beschreibungsklassen
für HomeMatic-Geräte von der Basisklasse erben.
Diese sind in der Regel nur mit Leseoperationen bestückt, um die Nutzung auf das gespeicherte
Mapping zu erleichtern.
Andere Klassen können Operationen hinzufügen, die die Daten aus der Dokumentation \cite{homematic_xmlrpc}
interpretieren und zusätzliche Informationen anbieten.

%\pagebreak

\section{Abbildung der Geräte}
\label{des_abbildung}

Um die HomeMatic-Geräte in den WieDAS-Datenraum einzupflegen oder es aus diesem zu erkenne,
benötigt es eine Abbildung.
Die grundlegenden Eigenschaften des Modells wurden in Abschnitt \ref{ana_abbildung}
festgelegt.
Diese Abbildung muss sowohl Geräteereignisse von HomeMatic in WieDAS bekannt geben,
als auch Ereignisse aus WieDAS aufnehmen und HomeMatic-Geräte entsprechend steuern.

\myfigure{h}{design_hm_dev}{Entwurf und Beziehungen eines HomeMatic-Geräts}

Abbildung \ref{abb_design_hm_dev} zeigt den Entwurf eines Geräts.
HomeMatic-Geräte werden mit einem oder mehreren Kanälen realisiert (Abschnitt \ref{gru_hm_obj}).
Ein Kanal hat mehrere Datenpunkte und eine feste Adresse, die beim Anmelden an der Zentrale
vergeben wird.
Diese werden mittels der \emph{Channel}-Klasse repräsentiert und besitzen nur
Informationen über die verfügbaren Datenpunkte und ihre Adresse.
Um die Kanäle zu kapseln, wird eine Geräte-Klasse entworfen, die das HomeMatic-Gerät
repräsentiert.
Diese Klasse ist abstrakt und kann von den verschiedenen Geräten vererbt werden.
In dieser Klasse werden für beide Richtungen Methoden hinzugefügt, die dann
entsprechend die WieDAS- oder HomeMatic-Ereignisse verarbeiten können.
% TODO: maybe?
%Bei der Abbildung wird auf das Generieren durch Meta-Elemente verzichtet, da
%die die Geräte zu unterschiedlich sind, um sie zu verallgemeinern (Abschnitt \ref{ana_abbildung}).
Das Basismodell der Geräte in der Klasse \emph{Device} besitzt lediglich eine Ansammlung
von Kanälen und eine Schnittstelle um auf AAL-Cache und HomeMatic-Ereignisse zu reagieren.
Geräte-Klassen besitzen genau wie Kanäle ein Attribut für die Adresse.
Durch die Speicherung der Adressen kann die Geräteverwaltung beim Durchsuchen mit einem
Adressenvergleich feststellen, welches Gerät für einen URI angesprochen werden muss.
Dabei hilft eine Methode, um den zusammengesetzten URI wieder in die Adresse zurückzuformen.
Sollte ein AAL-Cache Ereignis, also ein WieDAS-Datum, ein HomeMatic-Gerät ansprechen, so muss
das Gerät dies erfahren.
Dazu wird eine Methode eingeführt, die eine Liste von URIs zurückgibt, die vom Connector
registriert werden müssen.
%Da die Geräte evtl. noch mit der Zentrale kommunizieren müssen, muss der XML-RPC-Client
%der Klasse bekannt sein.

Das Identifikatoren-Mapping zwischen URI und Adresse für die Geräte
wird mittels einer einfachen Abbildungsregel realisiert.
Der URI wird aus dem Namen des WieDAS-Datenobjekts und der Adresse generiert, so
dass beide Identifikatoren gebündelt gespeichert werden können.
Im AAL-Cache sind die Geräte Datenstrukturen, die aus der IDL-Datei spezifiziert werden (Abschnitt \ref{gru_wiedas_daten}).
In WieDAS werden Geräte mit einer ID versehen, die von einem 16-Bit-Typ repräsentiert
werden (Abschnitt \ref{gru_wiedas_daten}).
Da die ID in den WieDAS-Datenobjekten frei verwendet werden kann
(Abschnitt \ref{ana_abbildung}), weil der URI als Identifikator dient, kann bei dem Setzen des Wertes
eine willkürliche Methodik angewandt werden.
Adressen in HomeMatic-Geräten sind Zeichenketten, wodurch keine 1:1 Abbildung
der jeweiligen Identifikatoren-Typen möglich ist.
Die Adressen, die bei ersten Versuchen üblich waren, hatten alle eine 7-stellige Nummer
am Ende der Adresse.
%Diese Adresse wird \emph{mod 65535} gerechnet, um eine ID für WieDAS zu erzeugen.
%Es wird eine mathematische Abbildung verwendet, um diese Adresse in der WieDAS-ID
%zu berücksichtigen.

\pagebreak

\myfigure{h}{design_abb}{HomeMatic-Geräteabbildung für den WieDAS-Datenraum}

Abbildung \ref{abb_design_abb} zeigt den Aufbau der Geräteabbildung für HomeMatic-Geräte.
Da HomeMatic-Geräte in der Lage sind mehrere WieDAS-Geräte zu realisieren,
müssen die Kanäle indiziert werden.
Jedoch ist es ebenfalls möglich, dass ein einzelner Kanal (z.B. ein Dimmer-Kanal)
mehrere Funktionalitäten bereitstellt (z.B. eine schaltbare Steckdose und ein dimmbares Licht).
Dadurch ist es möglich, dass ein Gerätekanal mehrere URIs anbietet.
Zusammengefasst besitzt ein Gerät im Aufbau mehrere Kanäle und Methoden zur Ereignisbehandlung.
Diese formen anhand des Kanalindex 
welche wiederum mehrere
URIs abhängig vom Kanal mit der dazugehörigen Adresse und dem Index des Kanals und den
angebotenen WieDAS-Funktionalitäten.
Der grundlegende Aufbau eines URI ist daher:
%\emph{\/\<WieDAS-Funktion\>\/\<Adresse\>-\<\#Index\>}
\emph{/\textless WieDAS-Funktion \textgreater/\textless Adresse \textgreater - \textless \#Index \textgreater}

\myfigure{h}{design_dimmer}{Ein Funk-Anschnitt-Dimmaktor von HomeMatic in WieDAS}

Ein Gerät welches auf 2 WieDAS-Funktionen abgebildet werden kann, ist der
Funk-Anschnitt-Dimmaktor (Abbildung \ref{abb_design_dimmer}) von HomeMatic,
da er eine Schaltbare Steckdose ist, dessen Ausgangsleistung gesteuert werden kann.
Dieser kann für WieDAS einen \emph{OnOffOutput} (also eine schaltbare Steckdose) und
ein \emph{DimmableLight} realisieren.
Ein URI für eine Ein-Aus-Funktionalität (\emph{OnOffFunctionality}) von WieDAS,
welche vom 1. Kanal mit der Adresse ABC123 angeboten wird, hat folgenden Aufbau:
\emph{/OnOffFunctionality/ABC123-1}

%\vfill
%\pagebreak

\section{AAL-Cache-Schnittstelle}
\label{des_aalcache}

\myfigurewidth{h}{design_aalcache}{Benutzung der AAL-Cache-Schnittstelle im Connector}

Die Abbildung \ref{abb_design_aalcache} zeigt, wie der Connector mit der AAL-Cache-Schnittstelle
und den für WieDAS spezifischen Komponenten interagiert.
Um den URI zu formen, benutzt der Connector die Namen der Funktionalitäten in der WieDAS-IDL-Datei.
Das von der Programmiersprache abhängige Interface, also die Strukturen der Daten, werden erst
im Gerät verwendet, sobald es ein Datum in den WieDAS-Datenraum einpflegen oder auslesen muss.
Der AAL-Cache besitzt eine Schnittstelle zum Anfordern von Tags, Einpflegen von Daten und Registrieren
einer Rückruf-Funktion (Abschnitt \ref{gru_aalcache}).

Wie in Abschnitt \ref{des_abbildung} erläutert wurde, verwenden HomeMatic-Geräte Kanäle, die adressiert
werden.
Diese Adressen dienen als Basis für die AAL-Cache URIs.
Die URIs werden dann angefordert bzw. gesammelt, wenn ein Ereignis im Connector dafür sorgt, dass Geräte
diese verarbeiten müssen.
Für jeden URI wird ein Tag angefordert (Abschnitt \ref{gru_aalcache}).
Die URIs mit den dazugehörigen Tags werden mittels der AAL-Cache Schnittstelle registriert, so dass bei einer
Änderung, also die Erneuerung eines Datums, der Connector darüber mit dem Aufruf einer Callback-Funktion
informiert wird.
Eines Neu-Registrierung am AAL-Cache wird nur bei Auftreten von Änderungen an den Geräten durchgeführt.
Der AAL-Cache stellt für die Registrierung ein opaken Datentypen, der die Registrierung speichert.
Eine einmalige Registrierung mit vielen Tags im Gegensatz zu einer Registrierung pro Tag ermöglicht
es dem AAL-Cache die Registrierungen speicherschonend zu hinterlegen.
Dies ist auch der Grund, warum Geräte sich nicht selbst im Cache registrieren.

Das Einpflegen eines WieDAS-Datenobjekts geschieht bei jedem Ereignis der Geräte, welches
von der HomeMatic-Zentrale übertragen wird.
Damit das Datenobjekt eingetragen wird, bedient sich der Connector der Funktion von den Geräteobjekten,
das gerätespezifische Datum in den Cache einzutragen.
Um dem Objekt die Möglichkeit zu geben die WieDAS-Funktionalität mit entsprechenden Daten zu füllen,
wird das komplette Ereignis an das Objekt übermittelt.

\section{Homematic-Komponenten}
\label{des_homematic}

\myfigure{h}{design_hm_teile}{Übersicht der HomeMatic-Komponenten}

Die XML-RPC-Beschreibung von HomeMatic \cite{homematic_xmlrpc} enthält Informationen, wie die
XML-RPC-Schnittstelle der HomeMatic-Zentrale angeboten wird und wie sie zu benutzen ist.
Der Entwurf für die Verarbeitung der HomeMatic-Komponenten in Abbildung \ref{abb_design_hm_teile}
wird in zwei Teile aufgeteilt.
Die Teilkomponente verwenden die Typenabbildung für XML-RPC und spezifische HomeMatic-Daten.
Diese Modellierung wurde in Abschnitt \ref{des_xmlrpc_abbildung} beschrieben.
Zu einem gibt es den Teil der Schnittstelle, welcher die Aufrufe zur HomeMatic-Zentrale abstrahiert
und den Logikprozess (in diesem Fall der Connector), welcher von der HomeMatic-Zentrale aufgerufen wird.

\subsection{Schnittstelle}
\label{des_hm_iface}

Die Schnittstelle wird in einer Klasse abstrahiert.
Sie wird so erstellt, dass sie in der Lage ist die im Dokument spezifizierten Funktionen anzubieten.
Für die derzeitige Verwendung wird nicht der volle Funktionsumfang benötigt.
Die Klasse wird jedoch vollständig entworfen, so dass zukünftige Applikationen oder
Erweiterungen Gebrauch von ihr machen können.
In der Klasse wird ein XML-RPC-Client hinterlegt.
Der Client leitet die Funktionsaufrufe aus dem Connector mit Hilfe des Clients an die
Zentrale weiter und liefert die Antwort.
Falls nötig, werden für die Parameter und Rückgabewerte entsprechende Typkonvertierungen in die
HomeMatic-spezifischen XML-RPC-Daten durchgeführt.

Die wesentlichen Funktionen die von der Schnittstelle benutzt werden sind:
\begin{enumerate}
\item Initialisierung
\item Anfordern von Gerätebeschreibungen
\item Setzen von Werten in Geräte-Datenpunkten
\end{enumerate}
Das Initialisieren ist notwendig, um der Zentrale den Aufenthaltsort der Logikschicht
mitzuteilen.
Um Geräte vom Connector aus zu steuern, muss der entsprechende Wert über die Schnittstelle
gesetzt werden.
Dafür werden lediglich die Adresse, der Name des Datenpunkts und der Wert benötigt.
Die möglichen Werte gehen aus der Dokumentation der Datenpunkte hervor.
Es existieren zusätzlich Methoden um Signalstärken, Dienstnachrichten und weiteres
zu erhalten.

Falls ein Gerät von der Zentrale aus geändert wird, kann so die Logikschicht
darüber informiert werden.
Da die vorhandenen Geräte in der Logikprozess gespeichert werden, kann die Zentrale
die Gerätebeschreibungen anfordern, um die intern gespeicherten Informationen mit
dem Logikprozess abzugleichen.
Dazu muss der Logikprozess die Liste der Gerätebeschreibungen entsprechend der
Funktionsaufrufe zur Modifizierung der Liste behandeln.

\subsection{Logik}
\label{des_hm_logic}
Der 2. Teil der HomeMatic-Komponenten, der Logikprozess für die HomeMatic-Zentrale, wird auf der
Seite der Anwendung realisiert um Ereignisse und Veränderungen in der Gerätemenge zu erhalten.
Die Logikschicht wird mit einer Klasse realisiert und besitzt minimal die Methoden, die
von der CCU-Schnittstelle aufgerufen werden.
Die 4 Methoden die von der Logikschnittstelle bereitgestellt werden müssen sind:
\begin{enumerate}
\item Auflisten der bekannten Geräte in der Logikschicht
\item Reagieren auf neue Geräte
\item Reagieren auf gelöschte Geräte
\item Reagieren auf Geräteaktualisierungen
\item Reagieren auf Geräteereignisse
\end{enumerate}
Nachdem die Logikschnittstelle bei der HomeMatic-Zentrale mit der Initialisierungsfunktion
bekanntgemacht worden ist, ruft die Zentrale die Methode \emph{newDevices} der Logikschnittstelle
auf, um alle angelernten Geräte der Logikschnittstelle bekannt zu geben.
Diese soll, wie in Abschnitt \ref{gru_hm_ccu} beschrieben, rudimentäre Informationen über die Geräte
speichern, so dass die Schnittstelle einen Abgleich machen kann.
Für den Entwurf muss die benötigte Bibliothek für das Anbieten von XML-RPC Funktionen
(Abschnitt \ref{ana_schnitt}) berücksichtigt werden.

%\pagebreak

\myfigurescaled[0.35]{h}{design_hm_logic}{Logikschnittstelle und XML-RPC-Bibliothek}

In Abbildung \ref{abb_design_hm_logic} wird der Entwurf der Logikschnittstelle gezeigt und welche
Komponenten der XML-RPC-Bibliothek im Entwurf mit berücksichtigt werden müssen.
Die Logik-Klasse wurde als abstrakte Klasse entworfen.
Dadurch ist es möglich die eigentliche XML-RPC-Logik für die HomeMatic-Zentrale vom Connector
zu entkoppeln.
Die ableitende Klasse kann dann entsprechen auf spezifische Ereignisse und Updates reagieren.
Dazu dienen die virtuellen Methoden der Klasse.
Die Logik-Klasse besitzt kein Attribut für die Schnittstelle an welcher es angemeldet wurde.
Wenn abgeleitete Klassen einen Teil der Schnittstelle für die Ereignisbehandlung benötigen,
muss dies dort eingefügt werden.
Es wird eine Auflistung aller bereitgestellten Methoden mit ihren Namen in einer \emph{Registry} gespeichert,
welche für die XML-RPC-Bibliothek in \emph{Server} benötigt wird.
Wird der Server veranlasst eine Methode auszuführen, so wird die passende registrierte Methode
aufgerufen.
In diesen werden dann die XML-RPC-Datentypen entsprechend der HomeMatic-Dokumentation umgewandelt.
Die Methoden-Klassen rufen dann die Methoden der Logik-Klasse mit der korrekten Signatur auf.
Die Logik-Klasse verarbeitet dann die Ereignisse und führt je nach Art des Ereignis ein \emph{on...}
Methodenaufruf durch.

%\pagebreak

\myfigurewidth{h}{design_hm_logic_seq}{Ablauf der Registrierung von Methoden}

Abbildung \ref{abb_design_hm_logic_seq} zeigt den Verlauf bei der Registrierung einer Methode.
Dazu wird eine spezifische Methode erstellt und ihr gleichzeitig eine Referenz auf
sich selbst mitgegeben, damit die Methode die Möglichkeit hat der Logik-Klasse einen
eingehenden Aufruf mitzuteilen.
Nachdem alle Methoden mit einen Namen registriert wurden, wird der Server erstellt und die komplette
Registrierungsentität an den Server übermittelt.
Dieser wartet nun auf eingehende XML-RPC-Requests und ruft die \emph{execute}-Methode der spezifischen
Methode auf, sobald ein Request mit dem Methodennamen eintrifft.
Das Methodenobjekt kümmert sich nun um die Umwandlung der Typen entsprechend der HomeMatic-Dokumentation.
Als letzter Schritt ruft das Methodenobjekt die korrekte Methode der Logik-Klasse mit der korrekten
Methodensignatur auf.

\section{Verwaltung der Geräte}
\label{design_geverw}

Der Gerätemanager ist dafür zuständig, die in Abschnitt \ref{des_abbildung} und Abschnitt \ref{des_aalcache}
beschriebenen Aufgaben zu übernehmen.
Er kapselt das Callback für AAL-Cache und ist für das Laden der Geräte und Geräteverknüpfungen zuständig.

\subsection{Gerätemanager}
\label{des_gemenge}

\myfigurewidth{h}{design_gemenge}{Aufbau des Gerätemanagers}

Abbildung \ref{abb_design_gemenge} zeigt den Aufbau des Gerätemanagers und die interessanten Verbindungen zu den
Bibliotheken, dem AAL-Cache und dem AAL-Cache spezifischen Logikprozess der HomeMatic-Schnittstelle.
Der Gerätemanager wird durch den Connector erzeugt und bleibt dauerhaft als Objekt vorhanden.
Erreicht den Logikprozess ein Aufruf der HomeMatic-Zentrale, so werden entsprechend des Aufrufs die Geräte
geladen, gelöscht oder aktualisiert.
Werden Geräte in der Logikschnittstelle hinzugefügt oder entfernt, so wird der Gerätemanager dazu aufgefordert
die Geräte zu laden/entladen und erneut alle URIs von Geräten zu registrieren, die registriert werden müssen.
Tritt ein Geräteereignis von HomeMatic auf, so wird in der Liste der Geräte geprüft, ob ein Gerät angesprochen wird.
Ist im AAL-Cache ein Ereignis mit den registrieren URIs aufgetreten, wird nur in der Liste der verfügbaren Geräte
durchsucht und nicht versucht ein Gerät zu laden, da keine Gerätebeschreibung aus dem WieDAS-Datenraum vorliegt.
Um ein Gerät zu laden, muss sich eine dynamische Bibliothek beim Gerätemanager anmelden.
Dazu ruft der Connector nach dem Laden der Bibliothek eine \emph{register}-Funktion in der
Gerätebibliothek auf, welche dann den eigenen Gerätelader für entsprechende Gerätetypen registriert.
Dies passiert mit dem Aufruf von \emph{registerLoader} im Gerätemanager.
Der Gerätemanager ist dann in der Lage, je nach Gerätetyp den passenden Gerätelader zur Erstellung
von Geräten zu nutzen.
Die Vernküpfungsbibliotheken werden in Verbindung mit den Gerätebibliotheken geladen.
Dies geschieht einmal beim Starten des Connectors und bei Beadarf über den Aufruf der Methode.
Verknüpfungsbibliotheken werden vom Gerätemanager abgefragt, welche URI-Paare (Sender, Empfänger)
von dieser Bibliothek behandelt werden können.
Der Gerätemanager wird dann, beim Abarbeiten eines AAL-Cache Ereignisses prüfen, ob eine URI
in den Paaren enthalten ist und dann die Verarbeitungsmethode der Verknüpfungsbibliothek aufrufen.
Diese muss dann das Datum für den fehlenden URI (Empfänger) aus dem AAL-Cache auslesen und die gewünschte
Logik mit der Verbindung herstellen.

\subsection{Generator für Geräte und Verknüpfungen}
\label{des_gen}
Da die Implementierung der Geräte und der Verknüpfungen stets dem gleichen Schema folgen, wurde ein Generator
entwickelt, der die Grund-Implementierung und ein Modell verarbeitet, um die endgültige Implementierung zu erstellen.
Da die Abbildung der Geräte stark gerätespezifisch ist, muss im Modell selbst Programmcode geschrieben werden.
Für die restlichen Eigenschaften des Geräts wurde eine einfache Beschreibungssyntax bei der Modellierung
gewählt.
Die Geräteverknüpfungen sind weniger komplex, benötigen dennoch selbst geschriebenen Programmcode.

\myfigurewidth{h}{design_modell}{Verarbeitung des Generators}
Abbildung \ref{abb_design_modell} zeigt die Komponenten des Generators.
Es wurden mehrere, bei Arbeiten mit Generatoren typische, Komponenten entwickelt.
Der Generator selbst benutzt das Modell, welches der Anwender geschrieben hat und prüft die Eingaben
und gibt Rückmeldung bei Fehlern.
Dann werden die Datenstrukturen und Funktionen generiert und in die Templates an entsprechender
Stelle eingefügt.
Die Templates enthalten nur statisch relevanten Programmcode und sind sehr schlank, da die Generierung
der Schnittstelle und Implementierung sehr dynamisch ist.

Die Beschreibungssyntax der Modelle ist sehr einfach gestrickt.
Es gibt eine Folge von Sektionen (z.B. \emph{channels}, \emph{loading}) und deren Inhalt.
Die Inhalte werden jeweils durch ein Einrückungszeichen vom Sektionsnamen getrennt.
Die Sektion endet dann bei Beginn einer neuen Sektion.
Je nach Art der Sektion kann auch eine leere Zeile die Sektion beenden (z.B. bei Auflistungen).
Für die Modellierung der Geräte sind folgende Eigenschaften relevant:
\begin{itemize}
\item Verwendete Kanaltypen
\item Angebotene Gerätetypen
\item Angebotene WieDAS-Funktionen
\item Zusätzlich benötigte Schnittstellen (optional)
\item Zusätzliche Klassenattribute (optional)
\item Konstruktur-Implementierung (optional)
\item HomeMatic-Ereignis-Implementierung
\item AAL-Cache-Ereignis-Implementierung (optional)
\end{itemize}

Die nötigen Eigenschaften einer Geräteverknüpfung fallen geringer aus:
\begin{itemize}
\item Liste der Sender-URIs pro Verbindungstyp
\item Implementierung der Verbindungstypen
\end{itemize}

\section{Connector}
\label{des_connector}
Der eigentliche Connector besitzt, durch den Einsatz mehrerer Module für spezielle Aufgaben, selbst wenig Funktionalität.
Das Modul ist dafür zuständig die Konfiguration zu lesen, die Geräteverwaltung und die Kommunikationskomponenten
zu erstellen.

\myfigurewidth{h}{design_connector_seq}{Ablauf des Connector-Start}
Abbildung \ref{abb_design_connector_seq} zeigt die Sequenz beim Starten des Connectors durch den AAL-Cache.

Der Eintrittspunkt des Connectors wird in einer Datenstruktur festgelegt, welche in der Konfiguration für alle
Connectoren benannt wird und vom AAL-Cache Programm zu Beginn ausgelesen wird.
In dieser Eintrittsfunktion wird dann die übergebene Konfiguration für den Connector verarbeitet.
Der Connector erstellt dann mit Hilfe der Einstellungen alle notwendigen Objekte und registriert den Logikprozess
bei der HomeMatic-Zentrale.
Der Logikprozess wartet dann zusammen mit dem Cache-Callback des Gerätemanagers auf eingehende Ereignisse.
