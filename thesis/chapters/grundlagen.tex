\chapter{Grundlagen}

\section{Ambient Assisted Living}

Ambient Assisted Living oder kurz AAL steht für eine Verschmelzung von neuen Technologien und
dem sozialen Umfeld.
Die Anzahl der älteren und alleinstehenden Menschen steigt stärker als die Anzahl der Jugendlichen
\cite[Innovationen für ein selbstbestimmtes Leben]{aal_deutschland}.
Diese Menschen können im Alltag bei der Benutzung von neuen Technologien unterstützt werden.
Das Ziel des AAL ist es, die Lebensqualität der Person in seinem sozialen Umfeld zu heben.
Nicht nur ältere und alleinstehende Menschen sollen von AAL profitieren.
Durch ständige Forschung in diesem Bereich entstehen neue Konzepte und Produkte, welche wiederum
für jede Generation wertvoll sind \cite{mtidw}.
