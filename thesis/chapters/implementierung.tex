\chapter{Implementierung}
\label{implementierung}

Die in diesem Kapitel wird beschrieben, wie das Design aus Kapitel \ref{design}
unter der Berücksichtigung der funktionalen und nicht-funtkionalen Anforderungen
aus Abschnitt \ref{ana_anforderungen} implementiert wird.
Die Implementierung wird prototypisch erstellt, um Tests durchzuführen und eine
Bewertung des Konzepts zuzulassen.
Dazu stehen verschiedene Geräte von HomeMatic und Geräte, die bereits im WieDAS-Datenraum
getestet wurden, verwendet.

\section{Entwicklungsumgebung}
\label{impl_entw}
Während der Entwicklung wurden ausschließlich Arbeitsplätze mit einem Ubuntu-Linux \cite{ubuntu}
benutzt.
Je nach Version des Systems wurden verschiedene Abhängigkeiten nachinstalliert oder Anpassungen
an den Bauanleitungen vorgenommen.
Die Implementierung wurde mit einem Compiler für 32-Bit Systeme erstellt und auf 32- und 64-Bit-Systemen
getestet.
Während der Entwicklung wurde das Versionskontrollwerkzeug Git \cite{git} genutzt, um bei der Entwicklung
feste Meilensteine zeitlich festzuhalten.
Um die ausführbare Datei, sowie die dynamischen Bibliotheken zu erstellen wurde GNU Make \cite{make} und
die GNU Compiler Collection \cite{gcc} verwendet.
Die Programmiersprache der Implementierung des Generators ist Python \cite{python} in der Version 2.
Diese wurde gewählt, da bereits Erfahrungen mit dem Schreiben von Generatoren aus anderen Projekten existieren
und Python-Interpreter durchaus auf vielen Systemen eingesetzt werden können.
So kann Generierung der Programmteile für ein Gerät oder einer Verknüpfung unabhängig vom Entwicklungssystem
geschehen.
Die Programmiersprache für den Connector ist C++11 \cite{c++11}.
Da C++ heutzutage in der Regel sehr gut durch Compiler optimiert werden kann und nur eine starke
Die nichtfunktionalen Anforderungen aus Abschnitt \ref{ana_anforderungen} wären bei
starker Verwendung von Templates oder sehr komplexen Objekthierarchien gegeben.
Diese Sprachelemente werden aber nur selten oder gar nicht verwendet.
Die neuere Version wurde deshalb benutzt, da sie in Bezug auf Sprachelemente und Werkzeuge aus
der Standardbibliothek mehr anbietet und lesbarer geworden ist.
Die folgenden Werkzeuge und Bibliotheken wurden verwendet:
\begin{itemize}
\item GNU Compiler Collection 4.8.1
\item Ubuntu Linux 13.10
\item Python 2.7.5
\item GNU Make 3.81
\item xmlrpc-c 1.16.33
\end{itemize}

Die Entwicklung wurde dem Design entsprechend in 3 Phasen durchgeführt.
Die 1. Phase umfasst das Implementieren der HomeMatic-Schnittstellen und der XML-RPC-Logik
aus den Abschnitten \ref{des_xmlrpc_abbildung} und \ref{des_homematic}.
In der 2. Phase wurde die Anbindung an den AAL-Cache implementiert (Abschnitt \ref{des_aalcache}
und \ref{des_connector}).
Die letzte Phase hat dann die Zwischenkomponenten entwickelt, also die Geräteverwaltung aus
Abschnitt \ref{des_gemenge} und die Geräteabbildung aus Abschnitt \ref{des_abbildung}
mit Bezug auf die Verwendung des Generators aus Abschnitt \ref{des_gen}.

\section{HomeMatic-Komponente}
\label{imp_hm}

In diesem Abschnitt wird erklärt, wie der Client und der Server im Connector implementiert wurde.
Die Kernaspekte umfassen das Umwandeln der Datentypen der Bibliothek und das Aufbauen und Abschicken
von Requests, sowie das Implementieren und Anbieten von Methoden.
Dabei wird erklärt wie die XML-RPC-Bibliothek benutzt werden muss, wie primitiven Datentypen für die
Nutzung mit der Bibliothek konvertiert werden und wie die Basis-Klasse zur Implementierung der spezifischen
HomeMatic-Datentypen implementiert wird.
Anschließend wird die Implementierung der Schnittstellenabstraktion und des Logikprozesses beschrieben.

\section{Typumwandlung und Basis-Klasse}
\label{imp_hm_typ}

Für die Verwendung der XML-RPC-Typen in C++, müssen die Typen umgewandelt werden.
Die Bibliothek arbeitet immer mit einem opaken Typen (\emph{value}, Abschnitt \ref{des_xmlrpc_abbildung}).
Ein Objekt diesen Types besitzt die Information, um welchen Typen es sich eigentlich handelt.
Es besitzt weiter Operationen, um zu Strukturen und Arrays hinzugefügt zu werden.
Diese Operationen arbeiten mit der C-Repräsentierung des Typs.
Von ihm leiten spezifische Typen ab, welche dann wiederum mehrere Operationen anbieten.
Die Hauptaufgabe der abgeleiteten Klassen ist es den opaken XML-RPC-Typen in C++-spezifische
Datentypen umzuwandeln oder aus ihnen erstellen zu lassen.
Es existieren 2 weitere Typen, um einen Fehler und eine Parameterliste zu repräsentieren.
Die Fehler-Klasse wird verwendet, um zu prüfen, ob ein Fehler bei einem Request aufgetaucht ist
oder einen Fehler dem Aufrufer (der HomeMatic-Zentrale) zurückzugeben.
Die Parameterliste wird benutzt, um in einer Methode die übergebenen Parameter zu extrahieren oder
für einen Request die Parameter zusammenzusetzen.

Für die Umwandlung des opaken Typen wurden dennoch einfache Funktionen geschrieben und für
Inline-Ersetzung deklariert.
Dadurch wird lediglich die Sichtbarkeit der Bibliothek vom Code getrennt.
Das Anbieten von Inline-Ersetzungen für Funktionen wurde durchgängig in der Implementierung für alle
Definitionen angewandt, wo die Definition der Subroutinen sehr schlank ist.
Das Gerüst für die in Abbildung \ref{des_xmlrpc_abbildung} dargestellte Basisklasse aller komplexeren
Datentypen, die über XML-RPC übertragen werden, wird fast ausschließlich lesend benutzt.
Der schreibende Zugriff auf das Objekt findet in den Geräteklassen statt, falls die intern
gespeicherten Kanäle aufgrund eines Ereignisses aktualisiert werden müssen und im Logikprozess,
um eine eigene Gerätebeschreibung mit der Adresse und der Version als Schlüssel (Abschnitt \ref{gru_hm_ccu})
für den Abgleich durch die Zentrale zuzulassen.
Die Modifizierungsoperation überschreibt den kompletten Assoziationseintrag, was je nach Verhalten
des Assoziationstyps in C++ eine sehr zeitintensive Operation sein kann.
Da die Operation jedoch selten benötigt wird, sind keine Optimierungen nötig.

\lstset{language=C++}
\begin{lstlisting}[frame=single,caption={Modifizieren eines Basisobjekts für die Übertragung mit XML-RPC},label=modify_base]
void Base::modify_map(char const *key, xmlrpc_c::value const &value)
{
  auto iterator = map_.find(key);
  std::pair<std::string, xmlrpc_c::value> const new_value = { key, value };

  if (iterator != map_.end())
  {
    // key already known
    // erase old and insert new
    iterator = map_.erase(iterator);
  }

  map_.insert(iterator, new_value);
}
\end{lstlisting}

Quellcode \ref{modify_base} zeigt die Modifizierungsoperation.
Zunächst wird eine leere Assoziation erstellt.
Dann werden alle Schlüssel aus der gespeicherten Assoziation gelesen und geprüft, ob dieser mit dem
Schlüssel übereinstimmt, der verändert werden möchte.
Falls ja, wird der alte Wert für diesen Schlüssel gelöscht und darauf das neue Schlüssel-Werte-Paar
eingefügt.
Da die Standardbibliothek ein Iterieren der Einträge zuließ, musste hier nicht wie befürchtet die
komplette Assoziation neu gespeichert werden.

\lstset{language=C++}
\begin{lstlisting}[frame=single,caption={Umwandung eines C++-Vektors in einein opaken XML-RPC-Typ},label=conv_from]
template <class V>
inline xmlrpc_c::value from_vector(std::vector<V> const &vector)
{
  std::vector<xmlrpc_c::value> result_vector;

  for (auto const &element: vector)
  {
    result_vector.push_back(from_type<V>(element));
  }

  return from_vector(result_vector);
}
\end{lstlisting}

In Quellcode \ref{conv_from} wird gezeigt, wie ein C++-Datentyp für die XML-RPC-Bibliothek zurückgewandelt wird.
Dazu müssen die komplexeren Strukturen in C++ behandelt werden und in XML-RPC-Typen umgewandelt werden, erst dann
wird der Container (z.B. Liste oder Mapping) in den XML-RPC-Typen umgewandelt.
Spezifische HomeMatic-Datentypen leiten von der Basisklasse ab und bieten nur lesende Zugriffsfunktionen auf die
gespeicherte Assoziation an.
Schreibzugriffe passieren wie beschrieben nur durch den Aufruf der Modifizierungsoperation der Basisklasse.
Bei diversen Testläufen wurde festgestellt, dass manche HomeMatic-Datentypen bestimmte Schlüssel immer
in den Assoziationen gesetzt haben, obwohl dies nicht weiter in der Dokumentation spezifiziert wurde.
In den betroffenen Klassen wurden beim Konstruktor Abfragen eingefügt, die prüfen ob diese Schlüssel einen Wert haben.
Falls dies nicht der Fall ist, schlägt die Konstruktion fehl.
Für alle anderen Klassen tritt beim Zugriff auf nicht vorhandene Schlüssel eine Ausnahme auf.

\section{Schnittstelle}
\label{imp_hm_iface}

Die Schnittstelle ist eine Klasse zur Abstraktion des XML-RPC-Clients (Abschnitt \ref{des_hm_iface}).
Sie besitzt die Methoden um die möglichen Funktionsaufrufe aus der HomeMatic-XML-RPC-Dokumentation \cite{homematic_xmlrpc}
zu implementieren.
Dabei wurden auch alle optional verfügbaren Methoden und Methoden implementiert, die für die prototypische
Implementierung nicht notwendig gewesen wären.
Falls in Zukunft weitere Geräte geschrieben werden, können diese den vollen Funktionsumfang der Zentrale verwenden.
Bei der Konstruktion eines Schnittstellenobjekts wird nur die XML-RPC-URL gespeichert und der Client der Bibliothek
erstellt.
Die XML-RPC-Bibliothek bietet eine Vielfalt von Clientmodellen für C++.
Bei den Modultests wurde herausgefunden, dass der zunächst verwendete Client die Anforderungen nicht erfüllt, da er keine
Fehlerbehandlung zuließ.
Im Fehlerfall traten im Client nur Ausnahmen auf, die zu wenig Information über die Fehlerart und Herkunft preisgaben.
Stattdessen wurde ein etwas komplexeres Clientmodell verwendet.
Für den Aufruf einer Funktion wurde eine Wrapperfunktion geschrieben, die eine Reihe von Parametern erhält und den Aufruf
entsprechend der Typen und der Anzahl der Parametern formt und an den Client übergibt.

\lstset{language=C++}
\begin{lstlisting}[frame=single,caption={Absetzen eines XML-RPC-Requests},label=xmlrpc_req]
/* transform arguments to xmlrpc_c::paramList and call the client */
template <class... Args>
xmlrpc_c::rpcOutcome call(xmlrpc_c::client_xml &client,
                          xmlrpc_c::carriageParm_http0 &carriage_param,
                          std::string const &method_name, Args &&... args);

std::string ccu_interface::getParamsetId(std::string const &address, std::string const &type)
{
  auto outcome = call(client_, carriage_param_, "getParamsetId", address, type);
  throw_if_fail(outcome);

  return types::to_string(outcome.getResult());
}
\end{lstlisting}

Quellcode \ref{xmlrpc_req} zeigt, wie die Parameter-Set-ID vom XML-RPC-Client angefordert wird.
Dabei werden die C++-Parameter übergeben und von der Typumwandlung aus dem vorherigen Abschnitt \ref{imp_hm_typ}
in die XML-RPC-Typen umgewandelt und an eine Parameterliste angehangen.
Diese wird dann mit der URL (\emph{carriage\_param\_}) und dem Methodennamen an den Client übergeben und
das Resultat zurückgegeben.
Dieses kann dann benutzt werden um eine genaue Fehleranalyse durchzuführen.
Im Normalfall werden Ausnahmen auftreten, sobald ein Fehler auftritt.
Im Falle eines Netzwerkfehlers oder eines Timeouts, treten spezifische Ausnahmen auf, die der Connector
dann abfangen kann, um diese zu ignorieren.
Dieses Verhalten führt dazu, dass der Connector, wie in der Analyse in Abschnitt \ref{ana_schnitt} entschieden wurde,
bei kleineren Netzwerkstörungen nicht abbricht.

\section{Logikprozess}
\label{imp_hm_logic}

Der Logikprozess muss implementiert werden, um auf Ereignisse der HomeMatic-Zentrale zu reagieren.
Der Aufbau und das Verhalten des Prozesses wurde in Abschnitt \ref{des_hm_logic} beschrieben.
Dieser Abschnitt zeigt, wie die Methoden beim Erstellen des Logikprozesses registriert werden
und wie eine Methode implementiert wird.

\lstset{language=C++}
\begin{lstlisting}[frame=single,caption={Erstellen des Objekts zum Reagieren auf HomeMatic-Ereignisse},label=hm_logic_create]
xmlrpc_c::serverAbyss::constrOpt ccu_logic::construct_server(unsigned short port)
{
 xmlrpc_c::serverAbyss::constrOpt constructor;

  constructor.registryPtr(registry_ptr_);
  constructor.portNumber(port);
  constructor.uriPath("/");

  return constructor;
}

ccu_logic::ccu_logic(unsigned short port)
  : registry_ptr_(create_registry()),
    server_(construct_server(port))
{
}
\end{lstlisting}

Quellcode \ref{hm_logic_create} zeigt wie das Objekt konstruiert wird.
Dabei wird zunächst ein Temporäres Konstruktorobjekt (\emph{constrOpt()}) erstellt, welches dann entsprechend
der gewünschten Konfiguration verändert wird.
Der Server lauscht auf allen IP-Adressen, die auf dem System konfiguriert worden sind und auf einem
spezifischen Port.
Falls dies ein Sicherheitsrisiko ist, muss das Konstruktorobjekt mit einem Socket des Betriebssystem
erstellt werden, welcher vorher an eine IP-Adresse gebunden wurde.
Alternativ kann das Betriebssystem eine Firewall einrichten, um vor falschen Zugriff zu schützen.
Die RPCs werden an der Dateisystemwurzel angeboten, dies weicht vom Standardpfad (\emph{\/RPC2\/}) ab.
Die Registrierung ist für die Speicherung der Methoden unter einem Namen zuständig.

\lstset{language=C++}
\begin{lstlisting}[frame=single,caption={Registrieren von Methoden im Logikprozess},label=hm_logic_cyc]
xmlrpc_c::registryPtr ccu_logic::create_registry()
{
  xmlrpc_c::registryPtr registry_ptr(new xmlrpc_c::registry);

  registry_ptr->addMethod("event", xmlrpc_c::methodPtr(new event_method(*this))
  ...

  return registry_ptr;
}
\end{lstlisting}

Um der registrierten Methode so wenig wie möglich Kontext zu geben, delegiert sie die Arbeit an den
Logikprozess.
Das führt dazu, dass eine gewollte zyklische Abhängigkeit beim Erstellen der Methode entsteht.
Quellcode \ref{hm_logic_cyc} zeigt wie die Methode zum Reagieren auf Ereignisse (\emph{event\_method})
registriert wird und gleichzeitig beim Erstellen eine Referenz auf das Objekt des Logikprozesses erhält.

\lstset{language=C++}
\begin{lstlisting}[frame=single,caption={Implementierung der Wrapper-Methoden im Logikprozess},label=hm_logic_meth]
template <class R, class M, class O, class... Args>
R call_handler(M method, O &object, Args &&... args)
{ 
  try
  {
    return (object.*method)(std::forward<Args>(args)...);
  }

  catch (std::exception const &exception)
  {
    throw xmlrpc_c::fault(exception.what());
  }
} 

deleteDevices_method::deleteDevices_method(ccu_logic &logic)
  : logic_(logic)
{
  //_signature = void(string, array);
  _help = "Gets called when devices are deleted.";
}

void deleteDevices_method::execute(xmlrpc_c::paramList const &paramList, xmlrpc_c::value *resultP)
{
  std::string const interface_id = paramList.getString(0);

  call_handler<void>(&ccu_logic::event, logic_, interface_id);

  // always set result, even in void, otherwise other side gets a fault
  *resultP = types::from_int(0);
}
\end{lstlisting}

Im Quellcode \ref{hm_logic_meth} wird die Implementierung der Wrapper-Methode zum Löschen von Ereignissen
gezeigt.
Sie besitzt eine Signatur, so dass bei der Introspektion des XML-RPC-Servers aufgelistet werden kann, welche
Parameter die Methode erhält und was sie zurück liefert.
Leider ist es nicht möglich einen \emph{Any}-Typen in die Signatur mit einzuschließen, wodurch die Signatur
nicht gesetzt wird und die Introspektion des Servers daher unzureichende Informationen Ergebnisse liefert.
Die \emph{execute}-Methode wird von der Bibliothek aufgerufen, um die eigentliche Arbeit der Methode
auszuführen.
Um sinnvolle Fehlermeldung zurückzugeben, wird die delegierte Arbeit in einem Try-Catch-Block verrichtet.
Dadurch werden die Fehler, die in der Verarbeitung auftreten können immer mit einer XML-RPC-Fault Antwort
enden.
Dies wurde gewählt, da nicht festgestellt werden konnte, ob und in welchem Fall die Bibliothek selbstständig
Fehler aus Ausnahmen erzeugt.
Auch wenn eine Methode, wie in diesem Fall, keinen Rückgabewert besitzt (\emph{void}), muss dieser trotzdem gesetzt
werden, da sonst eine Fault-Antwort generiert wird.

Das eigentliche Abarbeiten des Requests wird dann im Objekt des Logikprozesses verrichtet.
Dazu ruft die Wrapper-Methode wie in Quellcode \ref{hm_logic_meth} gezeigt die entsprechende Methode des Logikobjekts
auf und übergibt die korrekten, vorher extrahierten, Parameter.

\lstset{language=C++}
\begin{lstlisting}[frame=single,caption={Rumpf einer Methode mit Locking im Logikprozess},label=hm_logic_spec]
void ccu_logic::event(std::string const &interface_id, std::string const &address,
                      std::string const &value_key, xmlrpc_c::value const &value)
{
  std::lock_guard<std::mutex> lock(hm_connector::mutex());
  event_data const event_data = { interface_id, address, value_key, value };

  on_event(event_data);
}
\end{lstlisting}

Quellcode \ref{hm_logic_spec} zeigt, dass die Methode ein Locking vornimmt.
Dies ist notwendig, da es passieren kann, dass der AAL-Cache das Callback zeitgleich mit einem Ereignis aus
der HomeMatic-Zentrale aufruft.
Durch die jedoch geringe Wahrscheinlichkeit, dass dies eintrifft und die nicht vorhandenen Echtzeit-Anforderungen
beim Abarbeiten der Ereignisse, wird ein globales Locking vorgenommen.
Sollte sich der Prozess bei der Verarbeitung eines Ereignisses aus dem AAL-Cache befinden, können keine
HomeMatic-Ereignisse abgearbeitet werden und umgekehrt.
Die Methode ruft als letztes eine virtuelle Methode auf.
Diese kann dann von vererbenden Klassen implementiert werden, um die Anwendungsspezifische Logik zu realisieren.
Die Basis-Klasse besitzt nur Funktionalität, die von HomeMatic gefordert wird.

\section{Geräteabbildung und Verknüpfungen}
\label{imp_abb}

Die Implementierung der Geräteabbildung ist zunächst relevant für den Code-Generator.
Dieser erzeugt eine Implementierung eines Geräts, welches später mit einer dynamischen Bibliothek geladen werden kann.
Erstellt und gespeichert werden Geräte vom Gerätemanager.
Die Methoden zum Verarbeiten der Ereignisse werden zum einen im Logikprozess und zum anderen in der registrierten
AAL-Cache Callback-Funktion des Gerätemanagers aufgerufen.
Dieser Abschnitt zeigt interessante Details zur Implementierung des Generators, des Gerätemanagers und der Geräte und
Verknüpfungen selbst.

\subsection{Code-Generator}
\label{imp_abb_gen}

Der Generator ist sehr einfach gestrickt und besitzt wenig Funktionalität.
Das Python-Programm wird mit nur einem Parameter aufgerufen, um den Name der Modellierungsdatei
zu bilden.
Der Name dient gleichzeitig zum Referenzieren im Connector-Code.
Der Generator wird in 2 Teile unterteilt.
Der Top-Down-Parser liest eine Modellierungsdatei.
In der Modellierungsdatei werden die gewünschten Sektionen untersucht, die dann
vom Generator in einer geeigneten Datenstruktur gespeichert werden.
Der Generator meldet Fehler, falls Daten fehlen, die zum erfolgreichen
Generieren einer Implementierung fehlen.
Der 2. Teil liest die Templates und generiert an den notwendigen Stellen Code.
Bei der Code-Generierung wurde darauf geachtet, dass die erstellte Schnittstelle
und Implementierung den Coding-Style der restlichen Dateien befolgt und gut lesbar ist.

\lstset{language={}}
\begin{lstlisting}[frame=single,caption={Ausschnitt des Templates für die Geräte-Implementierung},label=gen_temp]
void registration()
{
  using namespace hm_connector;

  auto \${lower_devname}_loader = std::make_shared<devices::\${lower_devname}_loader>();

  try
  {
\${loading}
  }
  [...]
}
\end{lstlisting}

\lstset{language=Python}
\begin{lstlisting}[frame=single,caption={Beispiel für die Code-Generierung des Generators},label=gen_code]
loading_list = []
for loading in data['loading_spec']:
  loading_template = '\t\tg_device_manager->register_loader("%s", %s_loader);'
  loading_list.append(loading_template % (loading, data['lower_devname']))
data['loading'] = '\n'.join(loading_list)
\end{lstlisting}

Quellcode \ref{gen_code} zeigt beispielhaft die Generierung der Registrierungs-Funktion mit Verwendung
des Templates aus Quellcode \ref{gen_temp}.Das \emph{lower\_devname} dient wie zuvor angesprochen der Namensgebung der Modellierungsdatei und dem Referenzieren
im Code.
In der Modellierungsdatei muss lediglich eine \emph{loading}-Sektion auftauchen, die die angebotenen Gerätetypen
auflistet.
In den Templates wurden einfache Platzhalter zum Einfügen von Text verwendet.
Auf die Verwendung von komplexeren Template-Engines wie Jinja \cite{jinja} oder ähnliches wurde aufgrund der Einfachheit
der Templates und der unnötigen Abhängigkeit verzichtet.

\subsection{Geräte}
\label{imp_abb_geräte}

In diesem Abschnitt wird die erzeugte Implementierung aus dem vorherigen Abschnitt beschrieben.
Die Geräte müssen die Ereignisse behandeln, die vom AAL-Cache oder von der HomeMatic-Zentrale an den Connector
mitgeteilt werden.
Sollen AAL-Cache-Ereignisse behandelt werden, muss der Connector die URIs anfordern können, bei denen das Gerät
informiert werden möchte.
Für die Ereignisbehandlung benötigt das Gerät Zugriff auf die WieDAS-Datenstrukturen der IDL-Datei.
Da die Syntax der IDL-Dateien Anlehnung an C-Syntax haben, wurde eine direkte Umwandlung mit einem Texteditor
ohne komplexe Konvertierungsprogramme vollzogen.
Dabei wurde die Methodik von \emph{CORBA IDL to C++ Mapping} \cite{corba_cpp_map} angewandt.
Die Eintrittsfunktion der Bibliothek (\emph{registration}) muss in einem Block deklariert werden, der mit \emph{extern "C"} eingeleitet wird.
Damit wird sichergestellt, dass das C++-Name-Mangling abbgestellt wird.
Dadurch kann der Gerätemanager nach dem Laden der Bibliothek das Symbol abrufen, ohne das ABI-spezifische Name-Mangling
zu berücksichtigen.
In der Registrierungsfunktion ruft die Bibliothek selbst wiederum die Registrierungsmethode des globalen Gerätemanagers
auf, um den Gerätelader für einen bestimmten Gerätetypen zu registrieren.
Wie bereits im vorherigen Abschnitt in Quellcode \ref{gen_temp} zu sehen, erstellt die Bibliothek beim Laden ein
referenzzählendes Zeigerobjekt und übergibt dieses zusammen mit dem Gerätetyp, der durch den Gerätelader geladen werden kann,
an den Gerätemanager.

Der Konstruktor des Geräts teilt dem Basis-Gerät mit, welche Kanaltypen gespeichert werden sollen.
Im Basis-Gerät holt sich der Konstruktor bei der HomeMatic-Zentrale die komplette Gerätebeschreibung für alle Kanäle ab.
Dann wird geprüft ob einer der Kanaltypen mit dem geforderten Kanaltyp übereinstimmt und besorgt dann das \emph{VALUES}
Parameter-Set für diesen Kanal von der Zentrale und speichert es ab.
Wie bereits in Abschnitt \ref{ana_abbildung} analysiert wurde, hat die ID im WieDAS-Datenraum wenig Bedeutung im
Connector.
Um trotzdem einen aussagekräftigen Wert zu finden, wird der numerische Teil der Adresse (z.B. 123 bei der Adresse \emph{JEQ0095433:1})
in der ID gespeichert.
Da diese jedoch nur 16-Bit groß ist, wird im Falle eines Überlaufs einfach der größte Wert (65535) gespeichert.
Weiterhin besitzt das Basis-Gerät die Möglichkeit die Adresse aus einem URI zu extrahieren (Abschnitt \ref{des_abbildung}).
Dabei wird nur der WieDAS-Teil des URI abgetrennt, dann das HM-Präfix und der Index-Suffix aus dem Restteil gelöscht.
Die Methode zur HomeMatic-Ereignisbehandlung im Basis-Gerät schaut zunächst in den gespeicherten \emph{VALUES} Parameter-Sets
der Kanäle, ob diese durch das Ereignis aktualisiert werden müssen und führt die Aktualisierung durch.
Erst dann wird die Ereignisbehandlung des spezifischen Geräts mit dem Index des aktualisierten Parameter-Sets aufgerufen.

\lstset{language=C++}
\begin{lstlisting}[frame=single,caption={Ausschnitt der Implementierung für die HomeMatic-Ereignisbehandlung eines Dimmers},label=ger_hm]
void trailing_edge_dimmer::hm_event_impl(ccu_logic::event_data const &event_data,
                                         channel_index_type channel_index)
{
  [...]
  auto const paramset = channels_[channel_index].paramset();
  double const old_level = types::to_double(paramset.get_from_map("LEVEL"));

  if (event_data.value_key == "OLD_LEVEL")
  {
    dimmable_event = false;
    onoff_event = true;
    [...]
  }

  [...]

  std::ostringstream temp_uri;
  aalc_metadata metadata;

  temp_uri << "/HM" << address_ << "-" << channel_index;

  AALC_WID_SET_CONNID(metadata.writer, connector_id());
  AALC_WID_SET_CLNTID(metadata.writer, 0);
  metadata.lastModified = 0;

  if (onoff_event)
  {
    WieDAS::OnOffOutput_dt output;
    auto const onoffoutput_uri = std::string("/OnOffOutput") + temp_uri.str();
    int const tag = aalc_getTag(onoffoutput_uri.c_str());

    output.id = get_wiedas_id();
    output.state = old_level == 0 ? WieDAS::on : WieDAS::off;

    if (::aalc_put(tag, &metadata, sizeof metadata, &output, sizeof output))
    {
      throw std::runtime_error("aalc_put in trailing_edge_dimmer failed.");
    }
  }

  [...]
}
\end{lstlisting}

Quellcode \ref{ger_hm} zeigt, wie die WieDAS-Funktionalität \emph{OnOffOutput} vom Gerät
in den Datenraum eingepflegt wird.
Dazu werden die Metadaten für AAL-Cache vorbereitet und mit dem aktuellen Zeitstempel versehen.
Dann wird die URI geformt und je nach Dimmzustand des letzten Wertes und des neuen Wertes
der entsprechende Status eingetragen.

\lstset{language=C++}
\begin{lstlisting}[frame=single,caption={Ausschnitt der Implementierung für die AAL-Cache-Ereignisbehandlung eines Dimmers},label=ger_aal]
void trailing_edge_dimmer::aalc_event_impl(std::string const &uri, int tag,
                                           channel_index_type channel_index)
{
  auto const event_address = address_from_uri(uri);
  auto const &channel = channels_[channel_index];
  std::string const onoffoutput_uri = "/OnOffOutput/HM";
  bool const onoff_event = uri.substr(0, onoffoutput_uri.length()) == onoffoutput_uri;

  // can only be OnOffOutput or DimmableLight according to registered URIs
  if (onoff_event)
  {
    WieDAS::OnOffOutput_dt onoff_output;

    double const old_level = types::to_double(channels_[channel_index].paramset().get_from_map("LEVEL"));

    if (aalc_getCopy(tag, &onoff_output, sizeof onoff_output, NULL, 0) < 0)
    {
      AALC_ERR("HM-Connector: aalc_getCopy failed.");
      return;
    }

    if (onoff_output.state == WieDAS::off && old_level > 0)
    {
      interface_.setValue(channel.address(), "LEVEL", types::from_double(0));
      [...]
    }
  }

  [...]
}
\end{lstlisting}

\subsection{Verknüpfungen}
\label{imp_verknüpfungen}
Quellcode \ref{ger_aal} zeigt im Gegensatz, wie ein AAL-Cache-Ereignis behandelt wird.
Dazu wird in der URI geprüft, welche WieDAS-Funktionalität aktualisiert wurde.
Dann wird der richtige Zweig abgearbeitet und zunächst das Datum aus dem WieDAS-Datenraum ausgelesen.
Entsprechend der gespeicherten Werte aus HomeMatic-Sicht (Parameter-Sets) und den übertragenen Werten
aus dem WieDAS-Datenraum wird über die Schnittstelle das HomeMatic-Gerät bei Bedarf aktualisiert.

Geräteverknüpfungen werden anhand ihres Typs unterschieden (z.B. Ein/Aus und Dimmung für einen Dimmer).
Die Typen werden in der Modellierung benannt und ihnen entsprechend die URIs zugewiesen.
Beim Laden der Bibliothek für eine Verknüpfung wird dann eine Sammlung von Zeigern auf verschiedene
Typen angefordert.
Dadurch kann eine einzelne Bibliothek eine große Anzahl von Verknüpfungen abdecken.
Dabei speichert das Basisobjekt der Verknüpfung lediglich die URIs der Sender, so dass beim Auftreten
eines Ereignisses aus dem AAL-Cache geprüft werden kann, ob der Sender für eine Verknüpfung vorgesehen ist.
Tritt ein solches Ereignis ein, wird das Tag des Senders von AAL-Cache angefordert und der abgeleiteten
Verknüpfungsklasse mitgeteilt, welche dann die für die Verknüpfung relevante Verarbeitung durchführt.

\lstset{language={}}
\begin{lstlisting}[frame=single,caption={Ausschnitt aus der Modellierung einer Verknüpfung},label=conn_mod]
uris:
	onoff_dimmer: /* connect key to dimmer for on/off */
		"/OnOffFunctionality/HMKEQ0059462-0"

	regulation_dimmer: /* connect key to dimmer for regulation */
		"/LightRegulationFunctionality/HMKEQ0059462-0"

onoff_dimmer:
	auto const receiver = "/OnOffOutput/HMJEQ0193620-0";
	auto const receiver_tag = aalc_getTag(receiver);

	if (receiver_tag == -1)
	{
		AALC_ERR("HM-Connector: aalc_getTag failed.");
		return;
	}

	WieDAS::OnOffFunctionality_dt onoff_func;
	WieDAS::OnOffOutput_dt onoff_output;
	aalc_metadata metadata;

	if (aalc_getCopy(sender_tag, &onoff_func, sizeof onoff_func, NULL, 0) < 0)
	[...]

	if (aalc_getCopy(receiver_tag, &onoff_output, sizeof onoff_output, &metadata, sizeof metadata) < 0)
	[...]

	AALC_WID_SET_CONNID(metadata.writer, connector_id());
	AALC_WID_SET_CLNTID(metadata.writer, 0);
	metadata.lastModified = 0;

	onoff_output.state = onoff_func.command == WieDAS::on ? WieDAS::on : WieDAS::off;

	std::printf("HM-Connector: Dimmer connection: state->%d\n", onoff_output.state);

	if (aalc_put(receiver_tag, &metadata, sizeof metadata, &onoff_output, sizeof onoff_output) < 0)
	[...]
\end{lstlisting}

\lstset{language=C++}
\begin{lstlisting}[frame=single,caption={Implementierung der Eintrittsfunktion eines Verknüpfungstyps},label=conn_entry]
std::vector<std::shared_ptr<hm_connector::device_connection>> get_connections()
{
  std::vector<std::string> const onoff_dimmer_uris = {
    "/OnOffFunctionality/HMKEQ0059462-0"
  };

  auto onoff_dimmer = std::make_shared<hm_connector::onoff_dimmer>(onoff_dimmer_uris);

  std::vector<std::string> const regulation_dimmer_uris = {
    "/LightRegulationFunctionality/HMKEQ0059462-0"
  };

  auto regulation_dimmer = std::make_shared<hm_connector::regulation_dimmer>(regulation_dimmer_uris);
  return { onoff_dimmer, regulation_dimmer };
}
\end{lstlisting}


Quellcode \ref{conn_entry} zeigt, wie die Implementierung für das Laden der Verknüpfungen aus der Modellierung
in \ref{conn_mod} aussieht.
Dabei wird nur eine Liste der URIs beim Erstellen der Verknüpfungsklasse mitgeteilt und die Liste
der Verknüpfungen zurückgegeben.
In der zweiten Sektion in Quellcode \ref{conn_mod} wird die Verknüpfung eines Tasters mit einem einzigen Aktor (Dimmer)
gezeigt.
Es wird zunächst das Tag für den Receiver-URI ermittelt und dann die Daten aus dem WieDAS-Datenraum für Sender
und Receiver ausgelesen.
Dann wird das Receiver-Datum modifiziert und zurück in den Cache geschrieben.

\subsection{Gerätemanager}
\label{imp_abb_gemenge}

Der Gerätemanager aus Abschnitt \ref{des_gemenge} ist für das Laden der Geräte- und Verknüpfungsbibliotheken zuständig.
Er bietet die Schnittstelle zu den Geräten für alle anderen Komponenten des Connectors (z.B. des Logikprozesses in
der HomeMatic-Komponente).
Er existiert als globaler Zeiger, um so die Komplexität aus dem Connector zu nehmen, die entstehen würde, wenn der Manager
als Referenz-Attribut in den Objekten präsent wäre und in den Methoden immer als Parameter mit übergeben werden müsste.
Der Gerätemanager ist durch seine höhere Komplexität gegenüber anderen Bestandteilen des Connectors, hat er die meisten
Attribute.
Es werden alle registrierten Gerätelader in einer Assoziation von Typ und Zeiger zum Lader gespeichert.
Auch werden alle geladenen Geräte selbst in einer Assoziation von Adresse und Gerätezeiger gespeichert.
Für beide Assoziationen wurde eine \emph{unordered\_map} aus der C++-Standardbibliothek verwendet.
Diese wird verwendet, um den Lookup zu beschleunigen, benötigt dadurch aber etwas mehr Speicher, durch das Mitspeichern
einer Hash-Funktion für den Schlüssel (hier: Zeichenkette).
Der Grund für die Speicherung von Zeigern im Gegensatz zu Kopien ist, so dass eine Assoziation von Kanaladresse zu Gerät
und eine von Gerätetyp zu Lader möglich ist.
Würde man nur ein Gerät oder einen Lader speichern, so müsste man bei jedem eintretenden Ereignis prüfen, ob die Lader
einen gewissen Gerätetypen laden können bzw. Geräte Kanäle mit einer bestimmten Adresse besitzen.
Die weiteren Attribute des Gerätemanagers sind:
\begin{itemize}
\item Handles für geladene Bibliotheken
\item Geräteverknüpfungen
\item Registerierungs-Handle für den AAL-Cache
\item Verzeichnisnamen für dynamische Bibliotheken
\item Referenz zur HomeMatic-Schnittstelle
\end{itemize}

Die Methode zum Laden von Geräten schaut zunächst, ob das Gerät schon geladen wurde.
Ist dies nicht der Fall wird der Gerätetyp von der HomeMatic-Schnittstelle für die entsprechende
Adresse gelesen.
Da der Gerätetyp oft eine Modellnummer als Suffix hat (z.B. \emph{HM-LC-Dim1T-Pl-2}) und die Gerätetypen,
die beim Registrieren des Loaders angegeben sind, diese nicht haben (z.B. \emph{HM-LC-Dim1T-Pl}) wird nur
dieser Teil des Gerätetyps vergleichen und bei Übereinstimmung der Gerätelader dazu aufgefordert
das Gerät zu laden.
Die \emph{load\_external\_libraries} Methode im Connector behandelt das Laden der externen Bibliotheken.
Darin wird zunächst die komplette Geräte- und Verknüpfungsliste gelöscht und die Handles
zu den Bibliotheken wieder freigegeben und anschließend entfernt.
Dann werden die Verzeichnisse mit den Bibliotheken nach Dateien mit der Endung \emph{.so} durchsucht (die
Endung für dynamische Bibliotheken oder ``shared objects'' in Linux) und versucht zu laden.
Dieser Teil ist einer der wenigen, welcher Betriebssystemabhängige Funktionen benutzt.
Darin werden dann je nach Bibliothekstyp die Funktion \emph{registration} oder \emph{get\_conncection}
aufgerufen und dann jeweils im Loader- oder Connection-Mapping gespeichert.
Anschließend werden die Subscriptions im AAL-Cache (also das Subscription-Set) aktualisiert.
Um alle notwendigen Tags zu erhalten, werden alle Verknüpfungen nach den URIs für die Sender gefragt und
die Geräte werden gefragt, welche URIs für sie registriert werden müssen.
Das \emph{aalcache\_callback} ist dafür zuständig auf Ereignisse aus dem AAL-Cache zu reagieren.
Die Methode ist eine globale Funktion und der übergebene Parameter ist eine Referenz auf den Gerätemanager selbst.
Durch die Klassen-Sichtbarkeit (also die Deklaration in der Klassen) ist das Callback in der Lage
auf private Attribute und Methoden der Klasse zuzugreifen.
Genau wie beim Logikprozess in Abschnitt \ref{imp_hm_logic} muss im Callback ein globales Locking vollzogen werden.
Dadurch kann während der Abarbeitung nicht mehr auf Ereignisse der HomeMatic-Zentrale reagiert werden.
Im Callback werden alle URIs zu den übertragenen Tags besorgt und dann die Verknüpfungen
und Geräte aufgefordert die URIs entsprechend zu verarbeiten.

\section{Connector}
\label{imp_conn}

Die Implementierung des Connectors legt die zuvor beschriebenen Komponenten an und initialisiert sie.
Dann wird der Connector in einen Idle-State geschickt und reagiert nur noch auf Ereignisse aus dem AAL-Cache
oder von der HomeMatic-Zentrale.

\lstset{language=C++}
\begin{lstlisting}[frame=single,caption={Ausschnitt der Eintrittsfunktion des Connectors},label=conn]
int start(int connector_id, char const *config)
{
  AALC_DBG("HM-Connector: Starting with ID %d.", connector_id);
  AALC_DBG("HM-Connector: Configuration: %s", config);

  [...]
  std::uint16_t logicport = 0;
  [...]

  init_map const initializers =
  {
    {
      "logicport", [&logicport](std::string const &param)
      {
        unsigned long port = std::stoul(param);

        if (port > 65535)
        {
          throw std::runtime_error("invalid logicport");
        }

        logicport = static_cast<std::uint16_t>(port);
      }
    }
  };

  parse_configuration(config, initializers);
  [...]

  std::shared_ptr<hm_connector::ccu_interface> ccu_interface(server_url.str());
  std::shared_ptr<hm_connector::aalcache_ccu_logic> aalcache_ccu_logic(logicport);
  hm_connector::g_device_manager.reset(
    new hm_connector::device_manager(
      ccu_interface,
      connections_directory, devices_directory));

  std::thread thread(run_server, aalcache_ccu_logic);

  ccu_interface.init(logic_url.str(), hm_connector::ccu_interface_id());

  hm_connector::g_device_manager->load_external_libraries();

  thread.join();

  return AALC_SUCCESS;
}
\end{lstlisting}

Quellcode \ref{conn} zeigt, wie der Connector gestartet wird.
Beim Eintritt in den Connector durch den AAL-Cache durchsucht ein Parser die Konfiguration nach relevanten
Konfigurationsparametern.
Diese werden dann genutzt um die HomeMatic-Schnittstelle und den Logikprozess zu erstellen, sowie
den globalen Gerätemanager-Zeiger zu initialisieren.
Bis zu diesem Zeitpunkt darf keine der Komponenten den Gerätemanager verwenden.
Erst nachdem der Gerätemanager erstellt ist, wird der Logikprozess in den Idle-Zustand versetzt, so
dass er neue Verbindungen annehmen und die Requests verarbeiten kann.
Dann wird der HomeMatic-Schnittstelle mitgeteilt, an welchem Port der Logikprozess lauscht.
Um direkt Ereignisse verarbeiten zu können wird nachdem der Logikprozess gestartet wurde ein
einmaliges Laden der externen Bibliotheken veranlasst und dann auf Ereignisse gewartet.
In der Implementierungsdatei befindet sich eine Datenstruktur, die einen Zeiger auf die Funktion enthält,
die beim Starten aufgerufen werden soll.
Weiterhin enthält die Datenstruktur die API-Version, gegen die programmiert wurde und einen Funktionszeiger
zu einer Funktion die aufgerufen wird, sobald ein Tag gelöscht wird.
Diese Funktionalität wird aber vom AAL-Cache Programm derzeit noch nicht unterstützt.

\lstset{language={}}
\begin{lstlisting}[frame=single,caption={Konfigurationseintrag des Connectors für den AAL-Cache},label=config]
../../hm_conn/libhomematic_conn.so homematic_conn serverhost=10.0.0.23 logichost=10.0.0.21 logicport=5555 load_libraries=true dyn-connections=../../hm_conn/dyn-connections dyn-devices=../../hm_conn/dyn-devices
\end{lstlisting}
Quellcode \ref{config} zeigt die vom AAL-Cache verwendete Konfigurationsdatei (\emph{connectors.conf}) mit dem
relevanten Eintrag, um den Connector beim Programmstart zu laden.
Es muss der Aufenthaltsort der Bibliothek, sowie die Datenstruktur und die Konfigurationsparameter in der Datei
niedergeschrieben werden.

\section{Debugging}
\label{imp_dbg}

Um während der Entwicklung Programme auf ihre Funktionsweise hin zu überprüfen, wurde der Debugger gdb \cite{gdb}
und der Speicherdebugger valgrind \cite{valgrind} benutzt.
Um mit diesen Programmen Informationen aus dem Kompilat zu erhalten, musste das Makro \emph{NDEBUG} bei der Übersetzung
undefiniert bleiben.
Weiterhin mussten sämtliche Optimisierungsparameter beim Compileraufruf entfernt werden.
Dadurch war ich in der Lage bei auftretenden Ausnahmen einen Stack-Trace des Programms zu untersuchen und somit
auf die Fehlerquelle stoßen.
Der Speicherdebugger, den ich bereits in privaten Projekten häufig verwendet habe, hat mich dabei unterstützt
Fehler bei der Verwendung und Freilassung von Speicherblöcken bzw. Variablen zu finden.
Da C++ durch das RAII-Verfahren selten Platz für solche Fehler anbietet, zeigten sich dort wenig Fehler.
Vor der Verwendung des globalen Lockings, hatte ich ein merkwürdiges Verhalten in manchen Situationen.
Auch ein Debugger konnte mir durch die Komplexität des Multi-Threadings nicht weiterhelfen.
Für das einfache Debugging konnte ich einfache Ausgabe-Statements in meinen Connector einfügen oder habe das vom
AAL-Cache bereitgestellte \emph{AALC\_DBG} Debugging-Makro verwendet.
Bei der Entwicklung wurde kein Test-First Prinzip angewendet, da ich mir nicht zugetraut habe dieses Softwareentwicklungsprinzip
in dem Zeitraum anzuwenden.
Dennoch wurden die Teilkomponenten mit wenigen Modultests auf ihre Funktionalität hin und das Zusammenspiel zwischen verschiedenen
Geräten im WieDAS-Datenraum geprüft.

\section{Testläufe}
\label{imp_test}

Die Tests werden in dieser Woche in der Hochschule durchgeführt.
Dieser Abschnitt, sowie die 2 folgenden Kapitel folgen am Wochenende.

\subsection{Testumgebung}
\label{imp_test_env}

\subsection{Funktionstest an verschiedenen Geräten in einer WieDAS-Umgebung}
\label{imp_test_dev}
